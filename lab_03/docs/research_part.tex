Проведём замеры процессорного времени, которое затрачивается каждым алгоритмом на сортировку массивов различной длины, и сравним полученные результаты.

\section{Характеристики ПК}
\qquadПри проведении замеров времени использовался компьютер, имеющий следующие характеристики:
\begin{itemize}
	\item OC - Windows 10 Pro
	\item Процессор - Inter Core i7 10510U (1800 МГц)
	\item Объём ОЗУ - 16 Гб
\end{itemize}

\section{Измерения}
\qquadДля проведения замеров процессорного времени использовались массивы длин N. \\
$N \in \left\lbrace 10, 50, 100, 500, 1000, 3000, 7000, 10000, 50000, 100000 \right\rbrace$.
Их содержимое генерируется случайным образом.

Каждый замер проводится 5 раз для получения более точного среднего результата.\\

В \hyperref[table_4_1]{таблице 4.1} и \hyperref[table_4_2]{таблице 4.2} представлены результаты замеров процессорного времени работы реализаций алгоритмов (в сек).

\begin{table}[ph] \label{table_4_1}
	\caption{Результаты измерений на размерах до 100 элементов}
	\centering
	\begin{tabular}{|c|c|c|c|}
		\hline
		Размер n&&&\\
		/    &10 &50 & 100 \\
		Алгоритм    &&&\\
		\hline
		Сортировка пузырьком & $1.762\cdot10^{-7}$ & $2.38\cdot10^{-6}$ & $9.066\cdot10^{-6}$ \\
		\hline
		Сортировка вставками & $1.56\cdot10^{-7}$ & $1.687\cdot10^{-6}$ & $6.387\cdot10^{-6}$ \\
		\hline
		Поразрядная сортировка & $1.875\cdot10^{-6}$ & $6.883\cdot10^{-6}$ & $1.328\cdot10^{-5}$ \\
		\hline
	\end{tabular}
\end{table}

\newpage

\begin{table}[ph!] \label{table_4_2}
	\caption{Результаты измерений на размерах до 100 000 элементов}
	\centering
	\begin{tabular}{|c|c|c|c|c|c|c|c|}
		\hline
		Размер n&&&&&&&\\
		/    &500 &1 000 & 3 000 & 7 000 & 10 000 & 50 000 & 100 000 \\
		Алгоритм    &&&&&&&\\
		\hline
		Сортировка & $2.58\cdot10^{-4}$ & $8.53\cdot10^{-4}$ & 0.0077 & 0.043 & 0.089 & 5.924 & 23.705\\
		пузырьком &&&&&&&\\
		\hline
		Сортировка & $1.66\cdot10^{-4}$ & $6.07\cdot10^{-4}$ & 0.0054 & 0.0295 & 0.060 & 1.68 & 7.417\\
		вставками &&&&&&&\\
		\hline
		Поразрядная & $6.452\cdot10^{-5}$ & $1.28\cdot10^{-4}$ & $3.8\cdot10^{-4}$ & $8.9\cdot10^{-4}$ & 0.0013 & 0.0064 & 0.0127\\
		сортировка &&&&&&&\\
		\hline
	\end{tabular}
\end{table}

Согласно полученным результатам можно сделать следующие \textbf{выводы}:
\begin{itemize}
	\item на массивах малого размера (до 100 элементов) поразрядная сортировка показывает результаты по времени примерно на порядок хуже, чем сортировки пузырьком и вставками;
	\item что касается алгоритма сортировки вставками на таких массивах, то среди рассматриваемых трёх алгоритмов у него лучшие показатели по времени;
	\item на массивах большего размера поразрядная сортировка эффективнее, по сравнению с двумя другими алгоритмами, на несколько порядков;
	\item на таких массивах хуже всего показатели по времени у сортировки пузырьком;
	\item сортировка вставками обрабатывает подобные массивы лучше, чем сортировка пузырьком, но значительно уступает по времени поразрядному алгоритму.
\end{itemize}

\section*{Вывод}
\addcontentsline{toc}{section}{Вывод}
\qquadБыли проведены замеры процессорного времени, и на основе полученных данных были составлены сравнительные таблицы, описывающие время, которое каждый из алгоритмов затрачивает на сортировку массива конкретной длины. В результате анализа получившихся таблиц были сделаны выводы, приведённые выше.
