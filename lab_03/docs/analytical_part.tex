В этом разделе будут поставлены цель и основные задачи лабораторной работы, также будут описаны выбранные для оценки трудоёмкости алгоритмы сортировки.\\

\textbf{Цель} данной работы: оценить трудоёмкость алгоритмов сортировки.\\

Для достижения поставленной цели необходимо решить ряд следующих \textbf{задач}:
\begin{enumerate}
\item[1)] дать математическое описание;
\item[2)] описать алгоритмы сортировки;
\item[3)] дать теоретическую оценку трудоёмкости алгоритмов;
\item[4)] реализовать эти алгоритмы;
\item[5)] провести замеры процессорного времени работы алгоритмов на материале серии экспериментов;
\item[6)] провести сравнительный анализ алгоритмов.
\end{enumerate}

\textbf{Задача сортировки} состоит в перестеновке (переупорядочивании) входной последовательности из $n$ чисел ${a_1, a_2, ..., a_n}$, так, чтобы выполнялось условие
\begin{equation}
	a_1' <= a_2' <= ... <= a_n'
\end{equation} (в случае сортировки по неубыванию). Аналогичным образом осуществляется сортировка по невозрастанию \cite{Kormen}.

\section{Пузырьковая сортировка}
\qquadПредставляет собой один из самых популярных алгоритмов сортировки. В его основе лежит многократная перестановка соседних элементов, нарушающих порядок сортировки \cite{Kormen}.

\section{Сортировка вставками}
\qquadСчитается простым алгоритмом сортировки. На каждом шаге алгоритма для очередного элемента находится подходящая позиция в уже отстортированной части массива и осуществляется вставка этого элемента. 

\section{Поразрядная сортировка}
\qquadТакже называется LSD-сортировкой (Least Significant Digit - по младшей цифре). В этом алгоритме массив несколько раз перебирается и элементы группируются в зависимости от того, какая цифра находится в определённом разряде. 

Сначала значения сортируются по единицам, затем по десяткам, сохраняя отсортированность по единицам внутри десятков, затем по сотням, сохраненяя отсортированность по десяткам и единицам внутри сотен и так далее.

После обработки всех разрядов массив становится упорядоченным.  

\section*{Вывод}
\addcontentsline{toc}{section}{Вывод}
\qquadОбозначены и сформулированы цель и задачи лабораторной работы. Дано краткое описание рассматриваемых алгоритмов, принципов их работы.