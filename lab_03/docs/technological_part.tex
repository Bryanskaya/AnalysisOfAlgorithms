В данном разделе будет проведена оценка трудоёмкости алгоритмов сортировки и проведены тесты, направленные на проверку корректности работы.

\section{Выбранный язык программирования}
\qquadДля выполнения этой лабораторной работы был выбран язык программирования C++, так как есть большой навык работы с ним и с подключаемыми библиотеками, которые также использовались для проведения тестирования и замеров. Тестирование реализованных функций производилось путём сравнения полученного результата с результатом обработки массива методом из стандартной библиотеки algorithm \cite{algorithm}.

Использованная среда разработки - Visual Studio \cite{Visual}.

\section{Листинг кода}
\qquadНиже представлены Листиги \ref{code1}, \ref{code2}, \ref{code3} функций, реализующих сортировки массива.
\begin{lstlisting}[label=code1, caption = Сортировка пузырьком]
#pragma optimize("", off)

void bubble_sort(array_t& arr, int n)
{
	double temp;
	int n_1 = n -1;
	
	for (int i = 0; i < n; i++)
		for (int j = 0; j < n_1 - i; j++)
			if (arr[j] > arr[j + 1])
			{
				temp = arr[j];
				arr[j] = arr[j + 1];
				arr[j + 1] = temp;
			}
}

#pragma optimize("", on)
\end{lstlisting}

\begin{lstlisting}[label=code2, caption = Сортировка вставками]
#pragma optimize("", off)

void insert_sort(array_t& arr, int n)
{
	double temp;
	int k;
	
	for (int i = 1; i < n; i++)
	{
		temp = arr[i];
		k = 0;
		
		while (arr[k] < temp)
			k += 1;
		
		for (int j = i; j > k; j--)
			arr[j] = arr[j - 1];
		
		arr[k] = temp;
	}
}
#pragma optimize("", on)
\end{lstlisting}

\begin{lstlisting}[label=code3, caption = Поразрядная сортировка]
#pragma optimize("", off)

int* create_digit(int n)
{
	int* arr = create_array(n);
	int elem = 1;
	
	for (int i = 0; i < n; i++, elem *= 10)
		arr[i] = elem;
	
	return arr;
}

int count_rank(int num)
{
	int count = 0;
	
	while (num > 0)
	{
		count++;
		num /= 10;
	}
	
	return count;
}

int analyze_arr(array_t arr, int n, int& flag)
{
	int min_elem = arr[0], max_elem = min_elem, elem;
	
	for (int i = 1; i < n; i++)
	{
		elem = arr[i];
		if (elem > max_elem)
			max_elem = elem;
		if (elem < min_elem)
			min_elem = elem;
	}
	
	if (min_elem < 0)
	{
		max_elem -= min_elem;
		flag = -min_elem;
		
		for (int i = 0; i < n; i++)
			arr[i] -= min_elem;
	}
	
	return count_rank(max_elem);
}

int get_digit(int num, int i, int* digit)
{
	num %= digit[i + 1];
	num /= digit[i];
	
	return num;
}

void lsd_sort(array_t& arr, int n)
{
	int flag = 0, count, temp, d;
	int rank = analyze_arr(arr, n, flag);
	int* degree_10 = create_digit(rank + 1);
	int num_fig[10];
	array_t temp_res = create_array(n), temp_copy;
	
	for (int i = 0; i < rank; i++)
	{
		for (int j = 0; j < 10; j++)
			num_fig[j] = 0;
		
		for (int j = 0; j < n; j++)
			num_fig[get_digit(arr[j], i, degree_10)]++;
		
		count = 0;
		
		for (int j = 0; j < 10; j++)
		{
			temp = num_fig[j];
			num_fig[j] = count;
			count += temp;
		}
		
		for (int j = 0; j < n; j++)
		{
			d = get_digit(arr[j], i, degree_10);
			temp_res[num_fig[d]] = arr[j];
			num_fig[d]++;
		}
		
		temp_copy = arr;
		arr = temp_res;
		temp_res = temp_copy;
	}
	
	free_array(&degree_10);
	free_array(&temp_res);
	
	if (flag)
		for (int i = 0; i < n; i++)
			arr[i] -= flag;
}

#pragma optimize("", on)
\end{lstlisting}

\section{Результаты тестов}
\qquadДля тестирования были написаны функции, проверяющие, согласно заготовкам выше, случаи. Выводы о корректности работы делаются на основе сравнения результатов.\\

\textbf{Все тесты пройдены успешно.} Сами тесты представлены ниже (Листинг \ref{code_test}).\\

\begin{lstlisting}[label=code_test, caption = Тесты]
bool sort_cmp(array_t a, int n)
{
	bool res = true;
	void(*func_arr[])(array_t&, int) = { bubble_sort, insert_sort, lsd_sort };
	
	array_t c = copy_array(a, n);
	sort(c, c + n);
	
	for (int i = 0; i < 3 && res; i++)
	{
		array_t temp_arr = copy_array(a, n);
		
		(*func_arr[i])(temp_arr, n);
		
		res = cmp_array(c, temp_arr, n);
		
		free_array(&temp_arr);
	}
	
	free_array(&c);
	
	return res;
}

// Размер массива равен 1
void test_size_1()
{
	int n = 1;
	
	for (int i = 0; i < 3; i++)
	{
		array_t a = random_fill_array(n);
		
		if (!sort_cmp(a, n))
		{
			cout << endl << __FUNCTION__ << " FAILED" << endl;
			free_array(&a);
			return;
		}
		
		free_array(&a);
	}
	
	cout << endl << __FUNCTION__ << " OK" << endl;
}

// Произвольные массивы различный длин
void test_std()
{
	int n[] = { 3, 5, 8, 10, 12 };
	
	for (int i = 0; i < sizeof(n)/sizeof(n[0]); i++)
	{
		array_t a = random_fill_array(n[i]);
		
		if (!sort_cmp(a, n[i]))
		{
			cout << endl << __FUNCTION__ << " FAILED" << endl;
			free_array(&a);
			return;
		}
		
		free_array(&a);
	}
	
	cout << endl << __FUNCTION__ << " OK" << endl;
}

// Уже отсортированные по неубыванию массивы
void test_sorted()
{
	int n = 30;
	
	array_t a = random_fill_array(n);
	
	sort(a, a + n);
	
	if (!sort_cmp(a, n))
	{
		cout << endl << __FUNCTION__ << " FAILED" << endl;
		free_array(&a);
		return;
	}
	
	free_array(&a);
	
	cout << endl << __FUNCTION__ << " OK" << endl;
}

// Уже отсортированные по невозрастанию массивы
void test_reverse_sorted()
{
	int n = 30;
	array_t a = random_fill_array(n);
	
	sort(a, a + n);
	reverse(a, a + n);
	
	if (!sort_cmp(a, n))
	{
		cout << endl << __FUNCTION__ << " FAILED" << endl;
		free_array(&a);
		return;
	}
	
	free_array(&a);
	
	cout << endl << __FUNCTION__ << " OK" << endl;
}

// Массив одинаковых элементов
void test_same_elements()
{
	int n = 30;
	array_t a = create_array(n);
	
	for (int i = 0; i < n; i++)
		a[i] = 102;
	
	if (!sort_cmp(a, n))
	{
		cout << endl << __FUNCTION__ << " FAILED" << endl;
		free_array(&a);
		return;
	}
	
	free_array(&a);
	
	cout << endl << __FUNCTION__ << " OK" << endl;
}

void run_tests()
{
	test_size_1();
	test_std();
	test_sorted();
	test_reverse_sorted();
	test_same_elements();
}
\end{lstlisting}

\section{Оценка трудоёмкости}
\qquadПроизведём оценку трудоёмкости приведённых алгоритмов. Рассмотрим сортировку массива  $A[N]$, где $N$ - размер массива. Пусть $R$ - максимальное количество разрядов числа, которое есть в этом массиве.

\subsection{Сортировка пузырьком}
$f_{bubble} = 2 + 2 + N(2 + 3 + \frac{N - 1}{2}(3 + 4 + $$\left[\begin{array}{ll}
	0, & $л.с.$\\
	2 + 4 + 3, & $х.с.$
\end{array} \right.$));\\
$f_{bubble} = 4 + N(5 + \frac{N - 1}{2}(7 + $$\left[\begin{array}{ll}
	0, & $л.с.$\\
	9, & $х.с.$
\end{array} \right.$)).\\

\textbf{Лучший случай} (отсортированный массив):\\
$f_{bubble} = 4 + N(5 + 3.5(N - 1)) = 3.5N^2 + 1.5N + 4$.\\

\textbf{Худший случай} (отсортированный в обратном порядке массив):\\
$f_{bubble} = 4 + N(5 + 8(N - 1)) = 8N^2 - 3N + 4$.\\

\subsection{Сортировка вставками}
$f_{insert} = 2 + N(2 + 2 + 1 + $$\left[\begin{array}{ll}
	2, & $обратно отсортированный массив$\\
	2 + (\frac{N + 1}{2})3, & $отсортированный массив$
\end{array} \right. + 2 + $\\

$\qquad+\left[\begin{array}{ll}
\frac{N + 1}{2}6, & $обратно отсортированный массив$\\
0, & $отсортированный массив$
\end{array} \right. + 2)$.\\

\textbf{Лучший случай} (отсортированный массив):\\
$f_{insert} = 2 + N(17 + \frac{N + 1}{2}) = 1.5N^2 + 12.5N + 2$.\\

\textbf{Худший случай} (отсортированный в обратном порядке массив):\\
$f_{insert} = 2 + N(14 + (N + 1)3) = 3N^2 + 14N + 2$.\\


\subsection{Поразрядная сортировка}
\qquad\textbf{Функция получения значения конкретного разряда}:\\
$f_{digit} = 3 + 2 = 5$.\\

\textbf{Функция создания массива степеней 10}:\\
$f_{create} = 2 + 2 + (R + 1)(3 + 2) = 4 + 5(R + 1) = 9 + 5R$.\\

\textbf{Функция посчёта количества разрядов числа}:\\
$f_{count} = 1 + 1 + R(1 + 1) = 2 + 2R$.\\

\textbf{Функция, нормализующая массив}:\\
$f_{analyze} = 2 + 1 + 2 + N(2 + 2 + 1 + $$\left[\begin{array}{ll}
	0, & $л.с.$\\
	1, & $х.с.$
\end{array} \right. + 1 + $$\left[\begin{array}{ll}
0, & $л.с.$\\
1, & $х.с.$
\end{array} \right. ) + 1 + $$\left[\begin{array}{ll}
0, & $л.с.$\\
1 + 1 + 2 + N(2 + 2), & $х.с.$
\end{array} \right. + + f_{count}$;\\
$f_{analyze} = 8 + 2R + N(6 + $$\left[\begin{array}{ll}
	0, & $л.с.$\\
	1, & $х.с.$
\end{array} \right. + $$\left[\begin{array}{ll}
0, & $л.с.$\\
1, & $х.с.$
\end{array} \right.) + $$\left[\begin{array}{ll}
0, & $л.с.$\\
4 + 4N, & $х.с.$
\end{array} \right.$.\\


\textbf{Функция, реализующая алгоритм поразрядной сортировки}:\\
$f_{lsd} = 1 + 1 + f_{analyze} + 1 + 1 + f_{create} + 2 + R(2 + 2 + 10(2 + 2) + 2 + N(2 + 3 + f_{digit}) + 1 + 2 + 10(2 + 5) + 2 + N(2 + 2 + f_{digit} + 4 + 2) + 3) + 1 + $$\left[\begin{array}{ll}
	0, & $л.с.$\\
	2 + N(2 + 2), & $х.с.$
\end{array} \right.$;\\
$f_{lsd} = 24 + 131R + 25RN + $$\left[\begin{array}{ll}
	0, & $л.с.$\\
	2 + 4N, & $х.с.$
\end{array} \right. + N(6 + $$\left[\begin{array}{ll}
0, & $л.с.$\\
1, & $х.с.$
\end{array} \right. + $$\left[\begin{array}{ll}
0, & $л.с.$\\
1, & $х.с.$
\end{array} \right.) + $$\left[\begin{array}{ll}
0, & $л.с.$\\
4 + 4N, & $х.с.$
\end{array} \right.$.\\

\textbf{Лучший случай} (массив состоит из неотрицательных элементов):\\
$f_{lsd} = 24 + 131R + 25RN + 6N$.\\

\textbf{Худший случай} (в массиве есть отрицательные элементы):\\
$f_{lsd} = 24 + 131R + 25RN + 2 + 4N + N(6 + 1 + 1) + 4 + 4N = 30 + 131R + 25RN + 16N$.\\

\section{Оценка времени}
\qquadПроцессорное время измеряется с помощью функции QueryPerformanceCounter библиотеки windows.h \cite{Query}. Осуществление замеров показано ниже (Листинг \ref{code_time}).
\begin{lstlisting}[label=code_time, caption = Замеры процессорного времени]
void test_range(vector<int>& n)
{
	for (int key : n)
	{
		cout << endl << endl << "Размер тестируемого массива: " << key << endl;
		
		cout << endl << "Сортировка пузырьком" << endl;
		test_time(bubble_sort, key);
		cout << endl << "Сортировка вставками" << endl;
		test_time(insert_sort, key);
		cout << endl << "Поразрядная сортировка" << endl;
		test_time(lsd_sort, key);
	}
}

void start_measuring()
{
	LARGE_INTEGER li;
	QueryPerformanceFrequency(&li);
	
	PCFreq = double(li.QuadPart) / 1000;
	
	QueryPerformanceCounter(&li);
	CounterStart = li.QuadPart;
}

double get_measured()
{
	LARGE_INTEGER li;
	QueryPerformanceCounter(&li);
	
	return double(li.QuadPart - CounterStart) / PCFreq;
}


void test_time(void(*f)(array_t&, int), int n)
{
	array_t a = random_fill_array(n);
	
	int num = 0;
	start_measuring();
	
	while (get_measured() < 3 * 1000)
	{
		f(a, n);
		num++;
	}
	
	double t = get_measured() / 1000;
	cout << "Выполнено " << num << " операций за " << t << " секунд" << endl;
	cout << "Время: " << t / num << endl;
	
	free_array(&a);
}
\end{lstlisting}

\section*{Вывод}
\addcontentsline{toc}{section}{Вывод}
\qquadТаким образом, рассчитана трудоёмкость алгоритмов, приведены листинги кода каждой из функций, реализующих эти алгоритмы, а также листинг тестовых функций, направленных на проверку корректности их работы.