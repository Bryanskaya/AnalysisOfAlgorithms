В ходе лабораторной работы была достигнута поставленная цель, а именно, оценена трудоёмкость трёх алгоритмов сортировки массивов (сортировка пузырьком, вставками и поразрядная сортировка).  \\

В процессе выполнения были решены все задачи. Описаны все рассматриваемые алгоритмы, дана теоретическая оценка трудоёмкости каждого. Все проработанные алгоритмы реализованы, кроме того, были проведены замеры процессорного времени работы на материале серии экспериментов и проведён сравнительный анализ, сделаны выводы.\\

По результатам замеров процессорного времени сделаны следующие заключения.
\begin{itemize}
	\item Применение поразрядной сортировки эффективно на массивах большого размера (от 100 элементов), в то время, как на меньших размерах этот алгоритм требует больших временных затрат по сравнению со другими рассматриваемыми алгоритмами.
	\item Чем больше размер массива, тем лучше показатели по времени демонстрирует поразрядная сортировка по сравнению с двумя другими алгоритмами.
	\item Сортировка пузырьком, уступает сортировке вставками на любых размерностях исследуемых массивов, что и ожидалось из оценки трудоёмкости.
	\item Среди трёх рассматриваемых алгоритмов сортировка вставками имеет лучшие временные показатели на массивах до 100 элементов, также может быть применима и на массивах, большего размера.
\end{itemize}
