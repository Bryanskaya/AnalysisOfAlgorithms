Рассмотрим выбранные алгоритмы сортировки. Для упрощения задачи будем сортировать последовательность по неубыванию. 
\section{Сортировка пузырьком}
\qquadОсуществляется проход по массиву от начала до конца, в процессе меняя местами неотсортированные соседние элементы.

В результате первого прохода на последнем месте окажется максимальный элемент. Далее снова делается проход по неотсортированной части массива (от первого до предпоследнего) и так же меняются неупорядоченные соседние элементы. Таким образом, на предпоследнее место будет помещён второй по величине элемент.

Действия повторяются до тех пор, пока не обработается вся неотсортированная часть. \\

\textbf{Схема} алгоритма представлена на Рис.\ref{fig1:image}.
%\begin{figure}[h]
%	\begin{center}
%		{\includegraphics[scale = 0.46]{schemes/bubble}}
%		\caption{Сортировка пузырьком}
%		\label{fig1:image}
%	\end{center}
%\end{figure}

\section{Сортировка вставками}
\qquadВ этом алгоритме рассматриваемый массив условно делится на две части: отсортированная и нет. 

В начале работы отсортированной частью считается нулевой элемент. Далее берётся каждый следующий и сравнивается с уже отсортированной частью. Находится подходящая для текущего элемента позиция в ней, осуществляется сдвиг уже отсортированных элементов, но больших по величине, чем рассматриваемый. И затем рассматриваемый элемент помещается на найденную позицию.

И так до тех пор, пока не просмотрится вся неотсортированная часть.\\ 

\textbf{Схема} алгоритма представлена на Рис.\ref{fig2:image}.
%\begin{figure}[h]
%	\begin{center}
%		{\includegraphics[scale = 0.46]{schemes/insert}}
%		\caption{Сортировка вставками}
%		\label{fig1:image}
%	\end{center}
%\end{figure}