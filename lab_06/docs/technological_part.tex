В данном разделе будут приведены листинги функций разрабатываемых алгоритмов.

\section{Выбранный язык программирования}
\qquadДля выполнения этой лабораторной работы был выбран язык программирования C++, так как есть большой навык работы с ним и с подключаемыми библиотеками, которые также использовались для проведения тестирования и замеров. \cite{c_plus_plus}

Использованная среда разработки - Visual Studio. \cite{Visual}

\section{Листинг кода}
\qquadНиже представлены Листиги \ref{code1}, \ref{code2} функций, реализующих алгоритмы решения задачи коммивояжёра.

\begin{lstlisting}[label=code1, caption = Алгоритм полного перебора]
matrix_t create_permutations(int n)
{
	vector<int> vct;
	matrix_t res;
	
	for (int i = 0; i < n; i++)
		vct.push_back(i);
	
	res.push_back(vct);
	while (next_permutation(vct.begin(), vct.end()))
		res.push_back(vct);
	
	return res;
}

bool way_is_exist(vector<int> vect, matrix_t& c)
{
	for (int i = 0; i < vect.size(); i++)
		if (c[vect[i]][vect[(i + 1) % vect.size()]] == 0)
			return false;
	
	return true;
}

int find_cost(vector<int> vect, matrix_t& c)
{
	int res = 0;
	
	for (int i = 0; i < vect.size(); i++)
		res += c[vect[i]][vect[(i + 1) % vect.size()]];
	return res;
}

void full_search(matrix_t& c, int n)
{
	vector<int> temp, tour;
	matrix_t prm = create_permutations(n);
	int num = prm.size(), min = 1e5, cost;
	
	for (int i = 0; i < num; i++)
	{
		temp = prm[i];
		if (way_is_exist(temp, c))
		{
			cost = find_cost(temp, c);
			
			if (cost < min)
			{
				min = cost;
				tour = temp;
			}
		}
	}
	
	cout << "Found tour: ";
	for (int i = 0; i < tour.size(); i++)
		cout << tour[i] << " ";
	cout << endl;
	
	cout << "Its length: " << min << endl;
}
\end{lstlisting}

\begin{lstlisting}[label=code2, caption = Муравьиный алгоритм]
matrix_double_t create_pheromone(int n)
{
	matrix_double_t res;
	vector<double> temp;
	
	for (int i = 0; i < n; i++)
	{
		for (int j = 0; j < n; j++)
		temp.push_back(INIT_PHR);
		res.push_back(temp);
	}
	
	return res;
}

double find_avg_path(matrix_t& c)
{
	int n = c.size();
	double res = 0;
	
	for (int i = 0; i < n; i++)
		for (int j = 0; j < n; j++)
			res += c[i][j];
	
	return res / n;
}

vector<ant_t> create_colony(int n)
{
	vector<ant_t> res;
	
	for (int i = 0; i < n; i++)
	{
		ant_t ant;
		ant.done_path.push_back(i);
		ant.cur_pos = i;
		ant.len_path = 0;
		
		res.push_back(ant);
	}
	
	return res;
}

bool is_exist(int index, vector<int> path)
{
	for (int i = 0; i < path.size(); i++)
		if (path[i] == index)
			return true;
	return false;
}

void update_ant(int new_pos, int len, ant_t& ant)
{
	ant.done_path.push_back(new_pos);
	ant.len_path += len;
	ant.cur_pos = new_pos;
}

int find_next_top(matrix_t& c, vector<int>& pos_path, ant_t& ant, matrix_double_t& phr, double a, double b)
{
	int k = 0;
	vector<double> p;
	double p_temp, sum = 0, cur_sum = 0, comp_p;
	
	for (int i = 0; i < pos_path.size(); i++)
	{
		int node = pos_path[i];
		if (c[ant.cur_pos][node])
			p_temp = pow(phr[ant.cur_pos][node], a) / pow(c[ant.cur_pos][node], b);
		else
			p_temp = 0;
		
		p.push_back(p_temp);
		sum += p_temp;
	}
	
	if (sum == 0)
		return -1;
	
	comp_p = (double)rand() / RAND_MAX * sum;
	
	while (cur_sum < comp_p)
	{
		cur_sum += p[k];
		k++;
	}
	
	return k-1;
}

void find_path(matrix_t& c, matrix_double_t& phr, ant_t& ant, double a, double b)
{
	vector<int> pos_path;
	int ind, flag = 1;
	
	for (int i = 0; i < c.size(); i++)
		if (i != ant.cur_pos)
			pos_path.push_back(i);
	
	while (pos_path.size() != 0)
	{
		ind = find_next_top(c, pos_path, ant, phr, a, b);
		if (ind == -1)
			break;
		
		update_ant(pos_path[ind], c[ant.cur_pos][pos_path[ind]], ant);
		
		pos_path.erase(pos_path.begin() + ind);
		if (pos_path.size() == 0 && flag)
		{
			pos_path.push_back(ant.done_path[0]);
			flag = 0;
		}
	}
}

void find_vaporization(matrix_double_t& phr, double k)
{
	for (int i = 0; i < phr.size(); i++)
		for (int j = 0; j < phr[i].size(); j++)
			phr[i][j] = (1 - k) * phr[i][j];
}

void increase_phr(matrix_double_t& pheromone, vector<ant_t> colony, double q)
{
	for (int i = 0; i < colony.size(); i++)
		for (int j = 0; j < colony[i].done_path.size() - 1; j++)
		{
			int node1 = colony[i].done_path[j];
			int node2 = colony[i].done_path[j+1];
			pheromone[node1][node2] += q / colony[i].len_path;
			if (pheromone[node1][node2] < MIN_K_PHR * INIT_PHR)
				pheromone[node1][node2] = MIN_K_PHR * INIT_PHR;
			pheromone[node2][node1] = pheromone[node1][node2];
		}
}

void elite_increase(double q, vector<int>& tour, int len, matrix_double_t& phr)
{
	int num_el_ant = 2;
	
	for (int i = 0; i < tour.size() - 1; i++)
	{
		int node1 = tour[i];
		int node2 = tour[i + 1];
		
		phr[node1][node2] += num_el_ant * q / len;
		phr[node2][node1] += num_el_ant * q / len;
	}
}

void ant_search(double a, double b, matrix_t& c, double k_vpr)
{
	double q = find_avg_path(c);
	matrix_double_t pheromone = create_pheromone(c.size());
	int t_max = 100, len = 1e7;
	vector<int> tour;
	
	for (int t = 0; t < t_max; t++)
	{
		vector<ant_t> colony = create_colony(c.size());
		
		for (int i = 0; i < colony.size(); i++)
		{
			find_path(c, pheromone, colony[i], a, b);
			if (colony[i].len_path < len && colony[i].done_path.size() == c.size() + 1)
			{
				len = colony[i].len_path;
				tour = colony[i].done_path;
			}
		}
		
		find_vaporization(pheromone, k_vpr);
		
		elite_increase(q, tour, len, pheromone);
		
		increase_phr(pheromone, colony, q);
	}
	
	cout << "Found tour: ";
	for (int i = 0; i < tour.size(); i++)
		cout << tour[i] << " ";
	cout << "\nIts length: " << len << endl;
}
\end{lstlisting}

\section{Результаты тестов}
\qquadДля тестирования были написаны функции, проверяющие, согласно заготовкам выше, случаи. Выводы о корректности работы делаются на основе сравнения результатов.\\

\textbf{Все тесты пройдены успешно.} Сами тесты представлены ниже (Листинг \ref{code_test}).\\

\begin{lstlisting}[label=code_test, caption = Тесты]
bool test(matrix_t& c)
{
	vector<int> res1, res2;
	int len1, len2;
	
	len1 = full_search(c, c.size(), res1);
	len2 = ant_search(0.5, 0.5, c, 100, res2);
	
	if (len1 == len2 || res1.size() == 0 && res2.size() == 0)
		return true;
	else
		return false;
}
void test_standart(matrix_t& c)
{
	cout << endl << __FUNCTION__;
	if (test(c))
		cout << ":\tPASSED\n";
	else
		cout << ":\tFAILED\n";
}

void test_size_2(matrix_t& c)
{
	cout << endl << __FUNCTION__;
	if (test(c))
		cout << ":\tPASSED\n";
	else
		cout << ":\tFAILED\n";
}

void test_no_solution(matrix_t& c)
{
	cout << endl << __FUNCTION__;
	if (test(c))
		cout << ":\tPASSED\n";
	else
		cout << ":\tFAILED\n";
}

void test_equal(matrix_t& c)
{
	cout << endl << __FUNCTION__;
	if (test(c))
		cout << ":\tPASSED\n";
	else
		cout << ":\tFAILED\n";
}

void run_tests()
{
	matrix_t c1 = { {0, 1, 2}, {1, 0, 3}, {2, 3, 0} };
	matrix_t c2 = { {0, 10}, {10, 0} };
	matrix_t c3 = { {0, 0, 17}, {0, 0, 0}, {17, 0, 0} };
	matrix_t c4 = { {0, 4, 4}, {4, 0, 4}, {4, 4, 0} };
	
	test_standart(c1);
	test_size_2(c2);
	test_no_solution(c3);
	test_equal(c4);
}	
\end{lstlisting}

\section{Оценка времени}
\qquadПроцессорное время измеряется с помощью функции QueryPerformanceCounter библиотеки windows.h. \cite{Query} Осуществление замеров показано ниже (Листинг \ref{code_time}).

\begin{lstlisting}[label=code_time, caption = Замеры процессорного времени]
	
\end{lstlisting}

\section*{Вывод}
\addcontentsline{toc}{section}{Вывод}
\qquadТаким образом, приведены листинги кода каждой из функций, реализующих алгоритмы решения задачи коммивояжёра, а также листинг тестовых функций, направленных на проверку корректности их работы.
