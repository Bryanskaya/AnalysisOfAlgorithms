В этой лабораторной работе будут рассматриваться два алгоритма, решающие задачу коммивояжёра. \\

\textbf{Задача о коммивояжёре} (travelling-salesman problem) является одной из знаменитых задач теории комбинаторики, была поставлена в 1934 году и заключается в поиске самого выгодного (минимального по стоимости) маршрута, проходящего строго по одному разу по всем приведённым городам с последующим возвратом в исходный город. Таким образом, выбор подходящего пути осуществялется среди гамильтоновых циклов.\\

\textbf{Гамильтонов цикл}  - это такой цикл (замкнутый путь), который проходит через каждую вершину ровно по одному разу.\\

Задача коммивояжёра имеет ряд практических применений, к примеру, она использовалась для составления маршрутов лиц, занимающихся выемкой монет из таксофонов. В этом случае, в качестве пунктов, которые нужно посетить, выступали места установки таксофонов, а стоимость -- время в пути между двумя точками. \\

Также она используется в задаче о сверлильном станке. Сверлильный станок изготавливает металлические листы с определённым количеством отверстий, координаты которых заранее известны. Нужно найти кратчайший путь через все отверстия, то есть наименьшее время, затрачиваемое на изготовление одной детали. \\

Для решения этой задачи есть несколько алгоритмов, в этой лабораторной работе будут рассмотрены: алгоритм полного перебора и муравьиный алгоритм.
