Проведём серию экспериментов для исследования параметризации для метода муравьиного алгоритма и сделаем замеры процессорного времени, которое затрачивается им на решение задачи коммивояжёра.

\section{Характеристики ПК}
\qquadПри проведении замеров времени использовался компьютер, имеющий следующие характеристики:
\begin{itemize}
	\item OC - Windows 10 Pro
	\item Процессор - Inter Core i7 10510U (1800 МГц)
	\item Объём ОЗУ - 16 Гб
\end{itemize}

\section{Содержание экспериментов}
\qquadБудут рассмотрены следующие случаи:
\begin{enumerate}
	\item матрица стоимостей размера $8 \times 8$, значения от 1 до 10;
	\item матрица стоимостей размера $8 \times 8$, значения от 1 до 1000;
	\item матрица стоимостей размера $8 \times 8$, значения от 1 до 10, из них 20 элементов  равны 0 (примерно 31.25\%);
\end{enumerate}

Длина получаемого маршрута представляет собой лучшее значение из найденных за 3 прогона алгоритма.

\section{Эксперимент 1}
\qquadБыла использована матрица \ref{matrix_1}.

\begin{scriptsize} 
	\begin{equation}\label{matrix_1}
		\left(
		\begin{array}{cccccccc}
			 0& 5& 5& 4& 8& 9& 4& 7\\
			5&  0& 9& 9& 5& 9& 3& 9\\
			5& 9&  0& 3& 2& 2& 6& 7\\
			4& 9& 3&  0& 2& 1& 6& 8\\
			8& 5& 2& 2&  0& 7& 5& 6\\
			9& 9& 2& 1& 7&  0& 3& 9\\
			4& 3& 6& 6& 5& 3&  0& 8\\
			7& 9& 7& 8& 6& 9& 8&  0\\
		\end{array}\right)
\end{equation} \end{scriptsize}

Метод полного перебора определил кратчайший маршрут и его эталонную длину, равную 30. По результатам проведённой параметризации составлена \hyperref[table_1]{таблица 4.1}.


\begin{longtable}{|c|c|c|c|c|}
	\caption{Результаты эксперимента №1}\label{table_1}\\
	\hline
	
	$\alpha$ & $\rho$ & Количество  & Длина & Разница с  \\
	&& поколений & маршрута & эталонным значением\\
	\hline
	\endfirsthead
	
	\hline
	$\alpha$ & $\rho$ & Количество  & Длина & Разница с  \\
	&& поколений & маршрута & эталонным значением\\
	\hline
	\endhead
	
	\hline
	\multicolumn{5}{c}{\textit{Продолжение на следующей странице}}
	\endfoot
	\hline
	\endlastfoot
	
	0& 0& 15& 31& 1 \\
	0& 0.1& 15& 30& 0 \\
	0& 0.2& 15& 30& 0 \\
	0& 0.3& 15& 31& 1 \\
	0& 0.4& 15& 32& 2 \\
	0& 0.5& 15& 30& 0 \\
	0& 0.6& 15& 31& 1 \\
	0& 0.7& 15& 30& 0 \\
	0& 0.8& 15& 31& 1 \\
	0& 0.9& 15& 30& 0 \\
	0& 1& 15& 30& 0 \\
	\hline
	
	0.1& 0& 15& 30& 0 \\
	0.1& 0.1& 15& 30& 0 \\
	0.1& 0.2& 15& 30& 0 \\
	0.1& 0.3& 15& 30& 0 \\
	0.1& 0.4& 15& 30& 0 \\
	0.1& 0.5& 15& 31& 1 \\
	0.1& 0.6& 15& 30& 0 \\
	0.1& 0.7& 15& 30& 0 \\
	0.1& 0.8& 15& 30& 0 \\
	0.1& 0.9& 15& 30& 0 \\
	0.1& 1& 15& 30& 0 \\
	\hline
	
	0.2& 0& 15& 31& 1 \\
	0.2& 0.1& 15& 30& 0 \\
	0.2& 0.2& 15& 30& 0 \\
	0.2& 0.3& 15& 30& 0 \\
	0.2& 0.4& 15& 30& 0 \\
	0.2& 0.5& 15& 30& 0 \\
	0.2& 0.6& 15& 30& 0 \\
	0.2& 0.7& 15& 30& 0 \\
	0.2& 0.8& 15& 30& 0 \\
	0.2& 0.9& 15& 30& 0 \\
	0.2& 1& 15& 30& 0 \\
	\hline
	
	0.3& 0& 15& 30& 0 \\
	0.3& 0.1& 15& 30& 0 \\
	0.3& 0.2& 15& 30& 0 \\
	0.3& 0.3& 15& 30& 0 \\
	0.3& 0.4& 15& 30& 0 \\
	0.3& 0.5& 15& 30& 0 \\
	0.3& 0.6& 15& 30& 0 \\
	0.3& 0.7& 15& 30& 0 \\
	0.3& 0.8& 15& 30& 0 \\
	0.3& 0.9& 15& 30& 0 \\
	0.3& 1& 15& 30& 0 \\
	\hline
	
	0.4& 0& 15& 30& 0 \\
	0.4& 0.1& 15& 30& 0 \\
	0.4& 0.2& 15& 31& 1 \\
	0.4& 0.3& 15& 30& 0 \\
	0.4& 0.4& 15& 30& 0 \\
	0.4& 0.5& 15& 30& 0 \\
	0.4& 0.6& 15& 30& 0 \\
	0.4& 0.7& 15& 30& 0 \\
	0.4& 0.8& 15& 30& 0 \\
	0.4& 0.9& 15& 30& 0 \\
	0.4& 1& 15& 31& 1 \\
	\hline
	
	0.5& 0& 15& 30& 0 \\
	0.5& 0.1& 15& 31& 1 \\
	0.5& 0.2& 15& 30& 0 \\
	0.5& 0.3& 15& 30& 0 \\
	0.5& 0.4& 15& 30& 0 \\
	0.5& 0.5& 15& 30& 0 \\
	0.5& 0.6& 15& 30& 0 \\
	0.5& 0.7& 15& 30& 0 \\
	0.5& 0.8& 15& 30& 0 \\
	0.5& 0.9& 15& 30& 0 \\
	0.5& 1& 15& 30& 0 \\
	\hline
	
	0.6& 0& 15& 30& 0 \\
	0.6& 0.1& 15& 30& 0 \\
	0.6& 0.2& 15& 30& 0 \\
	0.6& 0.3& 15& 30& 0 \\
	0.6& 0.4& 15& 30& 0 \\
	0.6& 0.5& 15& 30& 0 \\
	0.6& 0.6& 15& 30& 0 \\
	0.6& 0.7& 15& 30& 0 \\
	0.6& 0.8& 15& 30& 0 \\
	0.6& 0.9& 15& 30& 0 \\
	0.6& 1& 15& 30& 0 \\
	\hline
	
	0.7& 0& 15& 31& 1 \\
	0.7& 0.1& 15& 30& 0 \\
	0.7& 0.2& 15& 30& 0 \\
	0.7& 0.3& 15& 30& 0 \\
	0.7& 0.4& 15& 30& 0 \\
	0.7& 0.5& 15& 30& 0 \\
	0.7& 0.6& 15& 30& 0 \\
	0.7& 0.7& 15& 30& 0 \\
	0.7& 0.8& 15& 30& 0 \\
	0.7& 0.9& 15& 30& 0 \\
	0.7& 1& 15& 30& 0 \\
	\hline
	
	0.8& 0& 15& 30& 0 \\
	0.8& 0.1& 15& 30& 0 \\
	0.8& 0.2& 15& 30& 0 \\
	0.8& 0.3& 15& 31& 1 \\
	0.8& 0.4& 15& 31& 1 \\
	0.8& 0.5& 15& 31& 1 \\
	0.8& 0.6& 15& 31& 1 \\
	0.8& 0.7& 15& 30& 0 \\
	0.8& 0.8& 15& 30& 0 \\
	0.8& 0.9& 15& 30& 0 \\
	0.8& 1& 15& 30& 0 \\
	\hline
	
	0.9& 0& 15& 30& 0 \\
	0.9& 0.1& 15& 30& 0 \\
	0.9& 0.2& 15& 30& 0 \\
	0.9& 0.3& 15& 30& 0 \\
	0.9& 0.4& 15& 30& 0 \\
	0.9& 0.5& 15& 31& 1 \\
	0.9& 0.6& 15& 30& 0 \\
	0.9& 0.7& 15& 31& 1 \\
	0.9& 0.8& 15& 30& 0 \\
	0.9& 0.9& 15& 30& 0 \\
	0.9& 1& 15& 30& 0 \\
	\hline
	
	1& 0& 15& 31& 1 \\
	1& 0.1& 15& 31& 1 \\
	1& 0.2& 15& 31& 1 \\
	1& 0.3& 15& 30& 0 \\
	1& 0.4& 15& 30& 0 \\
	1& 0.5& 15& 30& 0 \\
	1& 0.6& 15& 32& 2 \\
	1& 0.7& 15& 31& 1 \\
	1& 0.8& 15& 31& 1 \\
	1& 0.9& 15& 30& 0 \\
	1& 1& 15& 30& 0 \\
	
\end{longtable}


\section{Эксперимент 2}
\qquadБыла использована матрица \ref{matrix_2}.

\begin{scriptsize} 
	\begin{equation}\label{matrix_2}
		\left(
		\begin{array}{cccccccc}
			  0& 178& 579& 264& 700& 853& 544& 96\\
			 178&  0& 388& 310& 294& 18& 103& 155\\
			 579& 388&  0& 471& 246& 423& 700& 725\\
			 264& 310& 471&  0& 621& 741& 503& 169\\
			 700& 294& 246& 621&  0& 514& 473& 180\\
			 853& 18& 423& 741& 514&  0& 519& 704\\
			 544& 103& 700& 503& 473& 519&  0& 909\\
			 96& 155& 725& 169& 180& 704& 909&  0\\
		\end{array}\right)
\end{equation} \end{scriptsize}

Метод полного перебора определил кратчайший маршрут и его эталонную длину, равную 1833. По результатам проведённой параметризации составлена \hyperref[table_2]{таблица 4.2}.


\begin{longtable}{|c|c|c|c|c|}
	\caption{Результаты эксперимента №2}\label{table_2}\\
	\hline
	
	$\alpha$ & $\rho$ & Количество  & Длина & Разница с  \\
	&& поколений & маршрута & эталонным значением\\
	\hline
	\endfirsthead
	
	\hline
	$\alpha$ & $\rho$ & Количество  & Длина & Разница с  \\
	&& поколений & маршрута & эталонным значением\\
	\hline
	\endhead
	
	\hline
	\multicolumn{5}{c}{\textit{Продолжение на следующей странице}}
	\endfoot
	\hline
	\endlastfoot
	
	0& 0& 15& 1833& 0 \\
	0& 0.1& 15& 1833& 0 \\
	0& 0.2& 15& 1833& 0 \\
	0& 0.3& 15& 1833& 0 \\
	0& 0.4& 15& 1833& 0 \\
	0& 0.5& 15& 1833& 0 \\
	0& 0.6& 15& 1833& 0 \\
	0& 0.7& 15& 1833& 0 \\
	0& 0.8& 15& 1833& 0 \\
	0& 0.9& 15& 1833& 0 \\
	0& 1& 15& 1833& 0 \\
	\hline
	
	0.1& 0& 15& 1833& 0 \\
	0.1& 0.1& 15& 1833& 0 \\
	0.1& 0.2& 15& 1833& 0 \\
	0.1& 0.3& 15& 1833& 0 \\
	0.1& 0.4& 15& 1833& 0 \\
	0.1& 0.5& 15& 1833& 0 \\
	0.1& 0.6& 15& 1833& 0 \\
	0.1& 0.7& 15& 1833& 0 \\
	0.1& 0.8& 15& 1833& 0 \\
	0.1& 0.9& 15& 1833& 0 \\
	0.1& 1& 15& 1833& 0 \\
	\hline
	
	0.2& 0& 15& 1833& 0 \\
	0.2& 0.1& 15& 1833& 0 \\
	0.2& 0.2& 15& 1833& 0 \\
	0.2& 0.3& 15& 1833& 0 \\
	0.2& 0.4& 15& 1833& 0 \\
	0.2& 0.5& 15& 1833& 0 \\
	0.2& 0.6& 15& 1833& 0 \\
	0.2& 0.7& 15& 1833& 0 \\
	0.2& 0.8& 15& 1833& 0 \\
	0.2& 0.9& 15& 1947& 114 \\
	0.2& 1& 15& 1833& 0 \\
	\hline
	
	0.3& 0& 15& 1833& 0 \\
	0.3& 0.1& 15& 1833& 0 \\
	0.3& 0.2& 15& 1833& 0 \\
	0.3& 0.3& 15& 1833& 0 \\
	0.3& 0.4& 15& 1833& 0 \\
	0.3& 0.5& 15& 1833& 0 \\
	0.3& 0.6& 15& 1833& 0 \\
	0.3& 0.7& 15& 1833& 0 \\
	0.3& 0.8& 15& 1833& 0 \\
	0.3& 0.9& 15& 1833& 0 \\
	0.3& 1& 15& 1833& 0 \\
	\hline
	
	0.4& 0& 15& 1833& 0 \\
	0.4& 0.1& 15& 1833& 0 \\
	0.4& 0.2& 15& 1833& 0 \\
	0.4& 0.3& 15& 1833& 0 \\
	0.4& 0.4& 15& 1833& 0 \\
	0.4& 0.5& 15& 1833& 0 \\
	0.4& 0.6& 15& 1833& 0 \\
	0.4& 0.7& 15& 1833& 0 \\
	0.4& 0.8& 15& 1833& 0 \\
	0.4& 0.9& 15& 1833& 0 \\
	0.4& 1& 15& 1833& 0 \\
	\hline
	
	0.5& 0& 15& 1833& 0 \\
	0.5& 0.1& 15& 1833& 0 \\
	0.5& 0.2& 15& 1833& 0 \\
	0.5& 0.3& 15& 1833& 0 \\
	0.5& 0.4& 15& 1833& 0 \\
	0.5& 0.5& 15& 1833& 0 \\
	0.5& 0.6& 15& 1833& 0 \\
	0.5& 0.7& 15& 1833& 0 \\
	0.5& 0.8& 15& 1833& 0 \\
	0.5& 0.9& 15& 1833& 0 \\
	0.5& 1& 15& 1833& 0 \\
	\hline
	
	0.6& 0& 15& 1833& 0 \\
	0.6& 0.1& 15& 1833& 0 \\
	0.6& 0.2& 15& 1833& 0 \\
	0.6& 0.3& 15& 1833& 0 \\
	0.6& 0.4& 15& 1833& 0 \\
	0.6& 0.5& 15& 1833& 0 \\
	0.6& 0.6& 15& 1833& 0 \\
	0.6& 0.7& 15& 1833& 0 \\
	0.6& 0.8& 15& 1833& 0 \\
	0.6& 0.9& 15& 1833& 0 \\
	0.6& 1& 15& 1833& 0 \\
	\hline
	
	0.7& 0& 15& 1833& 0 \\
	0.7& 0.1& 15& 1833& 0 \\
	0.7& 0.2& 15& 1947& 114 \\
	0.7& 0.3& 15& 1833& 0 \\
	0.7& 0.4& 15& 1833& 0 \\
	0.7& 0.5& 15& 1833& 0 \\
	0.7& 0.6& 15& 1833& 0 \\
	0.7& 0.7& 15& 1833& 0 \\
	0.7& 0.8& 15& 1947& 114 \\
	0.7& 0.9& 15& 1833& 0 \\
	0.7& 1& 15& 1833& 0 \\
	\hline
	
	0.8& 0& 15& 1833& 0 \\
	0.8& 0.1& 15& 1833& 0 \\
	0.8& 0.2& 15& 1833& 0 \\
	0.8& 0.3& 15& 1833& 0 \\
	0.8& 0.4& 15& 1833& 0 \\
	0.8& 0.5& 15& 1833& 0 \\
	0.8& 0.6& 15& 1833& 0 \\
	0.8& 0.7& 15& 2028& 195 \\
	0.8& 0.8& 15& 1833& 0 \\
	0.8& 0.9& 15& 2028& 195 \\
	0.8& 1& 15& 1833& 0 \\
	\hline
	
	0.9& 0& 15& 1833& 0 \\
	0.9& 0.1& 15& 1833& 0 \\
	0.9& 0.2& 15& 1833& 0 \\
	0.9& 0.3& 15& 1833& 0 \\
	0.9& 0.4& 15& 1833& 0 \\
	0.9& 0.5& 15& 1833& 0 \\
	0.9& 0.6& 15& 1833& 0 \\
	0.9& 0.7& 15& 1833& 0 \\
	0.9& 0.8& 15& 1833& 0 \\
	0.9& 0.9& 15& 1833& 0 \\
	0.9& 1& 15& 1833& 0 \\
	\hline
	
	1& 0& 15& 1833& 0 \\
	1& 0.1& 15& 1947& 114 \\
	1& 0.2& 15& 1833& 0 \\
	1& 0.3& 15& 2161& 328 \\
	1& 0.4& 15& 1833& 0 \\
	1& 0.5& 15& 1947& 114 \\
	1& 0.6& 15& 1833& 0 \\
	1& 0.7& 15& 1833& 0 \\
	1& 0.8& 15& 1833& 0 \\
	1& 0.9& 15& 1947& 114 \\
	1& 1& 15& 1833& 0 \\

\end{longtable}

\section{Эксперимент 3}
\qquadБыла использована матрица \ref{matrix_3}.

\begin{scriptsize} 
	\begin{equation}\label{matrix_3}
		\left(
		\begin{array}{cccccccc}
			0& 8& 2& 0& 1& 1& 3& 0\\
			8&  0& 0& 4& 2& 3& 4& 6\\
			2& 0&  0& 4& 0& 3& 1& 5\\
			0& 4& 4&  0& 9& 4& 0& 2\\
			1& 2& 0& 9&  0& 5& 3& 1\\
			1& 3& 3& 4& 5&  0& 6& 0\\
			3& 4& 1& 0& 3& 6&  0& 8\\
			0& 6& 5& 2& 1& 0& 8&  0\\
		\end{array}\right)
\end{equation} \end{scriptsize}

Метод полного перебора определил кратчайший маршрут и его эталонную длину, равную 17. По результатам проведённой параметризации составлена \hyperref[table_3]{таблица 4.3}.


\begin{longtable}{|c|c|c|c|c|}
	\caption{Результаты эксперимента №3}\label{table_3}\\
	\hline
	
	$\alpha$ & $\rho$ & Количество  & Длина & Разница с  \\
	&& поколений & маршрута & эталонным значением\\
	\hline
	\endfirsthead
	
	\hline
	$\alpha$ & $\rho$ & Количество  & Длина & Разница с  \\
	&& поколений & маршрута & эталонным значением\\
	\hline
	\endhead
	
	\hline
	\multicolumn{5}{c}{\textit{Продолжение на следующей странице}}
	\endfoot
	\hline
	\endlastfoot
	
	0& 0& 15& 18& 1 \\
	0& 0.1& 15& 18& 1 \\
	0& 0.2& 15& 17& 0 \\
	0& 0.3& 15& 17& 0 \\
	0& 0.4& 15& 18& 1 \\
	0& 0.5& 15& 17& 0 \\
	0& 0.6& 15& 17& 0 \\
	0& 0.7& 15& 17& 0 \\
	0& 0.8& 15& 17& 0 \\
	0& 0.9& 15& 17& 0 \\
	0& 1& 15& 17& 0 \\
	\hline
	
	0.1& 0& 15& 18& 1 \\
	0.1& 0.1& 15& 17& 0 \\
	0.1& 0.2& 15& 17& 0 \\
	0.1& 0.3& 15& 17& 0 \\
	0.1& 0.4& 15& 17& 0 \\
	0.1& 0.5& 15& 17& 0 \\
	0.1& 0.6& 15& 17& 0 \\
	0.1& 0.7& 15& 17& 0 \\
	0.1& 0.8& 15& 17& 0 \\
	0.1& 0.9& 15& 17& 0 \\
	0.1& 1& 15& 17& 0 \\
	\hline
	
	0.2& 0& 15& 17& 0 \\
	0.2& 0.1& 15& 17& 0 \\
	0.2& 0.2& 15& 17& 0 \\
	0.2& 0.3& 15& 17& 0 \\
	0.2& 0.4& 15& 17& 0 \\
	0.2& 0.5& 15& 17& 0 \\
	0.2& 0.6& 15& 17& 0 \\
	0.2& 0.7& 15& 17& 0 \\
	0.2& 0.8& 15& 17& 0 \\
	0.2& 0.9& 15& 17& 0 \\
	0.2& 1& 15& 17& 0 \\
	\hline
	
	0.3& 0& 15& 17& 0 \\
	0.3& 0.1& 15& 17& 0 \\
	0.3& 0.2& 15& 17& 0 \\
	0.3& 0.3& 15& 17& 0 \\
	0.3& 0.4& 15& 17& 0 \\
	0.3& 0.5& 15& 17& 0 \\
	0.3& 0.6& 15& 17& 0 \\
	0.3& 0.7& 15& 17& 0 \\
	0.3& 0.8& 15& 17& 0 \\
	0.3& 0.9& 15& 17& 0 \\
	0.3& 1& 15& 17& 0 \\
	\hline
	
	0.4& 0& 15& 17& 0 \\
	0.4& 0.1& 15& 17& 0 \\
	0.4& 0.2& 15& 17& 0 \\
	0.4& 0.3& 15& 17& 0 \\
	0.4& 0.4& 15& 17& 0 \\
	0.4& 0.5& 15& 17& 0 \\
	0.4& 0.6& 15& 17& 0 \\
	0.4& 0.7& 15& 17& 0 \\
	0.4& 0.8& 15& 17& 0 \\
	0.4& 0.9& 15& 17& 0 \\
	0.4& 1& 15& 18& 1 \\
	\hline
	
	0.5& 0& 15& 17& 0 \\
	0.5& 0.1& 15& 17& 0 \\
	0.5& 0.2& 15& 17& 0 \\
	0.5& 0.3& 15& 17& 0 \\
	0.5& 0.4& 15& 17& 0 \\
	0.5& 0.5& 15& 17& 0 \\
	0.5& 0.6& 15& 17& 0 \\
	0.5& 0.7& 15& 17& 0 \\
	0.5& 0.8& 15& 17& 0 \\
	0.5& 0.9& 15& 17& 0 \\
	0.5& 1& 15& 17& 0 \\
	\hline
	
	0.6& 0& 15& 17& 0 \\
	0.6& 0.1& 15& 17& 0 \\
	0.6& 0.2& 15& 17& 0 \\
	0.6& 0.3& 15& 17& 0 \\
	0.6& 0.4& 15& 17& 0 \\
	0.6& 0.5& 15& 17& 0 \\
	0.6& 0.6& 15& 17& 0 \\
	0.6& 0.7& 15& 17& 0 \\
	0.6& 0.8& 15& 17& 0 \\
	0.6& 0.9& 15& 17& 0 \\
	0.6& 1& 15& 17& 0 \\
	\hline
	
	0.7& 0& 15& 18& 1 \\
	0.7& 0.1& 15& 17& 0 \\
	0.7& 0.2& 15& 17& 0 \\
	0.7& 0.3& 15& 17& 0 \\
	0.7& 0.4& 15& 17& 0 \\
	0.7& 0.5& 15& 17& 0 \\
	0.7& 0.6& 15& 17& 0 \\
	0.7& 0.7& 15& 17& 0 \\
	0.7& 0.8& 15& 17& 0 \\
	0.7& 0.9& 15& 17& 0 \\
	0.7& 1& 15& 17& 0 \\
	\hline
	
	0.8& 0& 15& 17& 0 \\
	0.8& 0.1& 15& 17& 0 \\
	0.8& 0.2& 15& 17& 0 \\
	0.8& 0.3& 15& 17& 0 \\
	0.8& 0.4& 15& 17& 0 \\
	0.8& 0.5& 15& 17& 0 \\
	0.8& 0.6& 15& 17& 0 \\
	0.8& 0.7& 15& 17& 0 \\
	0.8& 0.8& 15& 17& 0 \\
	0.8& 0.9& 15& 17& 0 \\
	0.8& 1& 15& 17& 0 \\
	\hline
	
	0.9& 0& 15& 17& 0 \\
	0.9& 0.1& 15& 17& 0 \\
	0.9& 0.2& 15& 18& 1 \\
	0.9& 0.3& 15& 18& 1 \\
	0.9& 0.4& 15& 17& 0 \\
	0.9& 0.5& 15& 17& 0 \\
	0.9& 0.6& 15& 17& 0 \\
	0.9& 0.7& 15& 17& 0 \\
	0.9& 0.8& 15& 17& 0 \\
	0.9& 0.9& 15& 17& 0 \\
	0.9& 1& 15& 18& 1 \\
	\hline
	
	1& 0& 15& 18& 1 \\
	1& 0.1& 15& 18& 1 \\
	1& 0.2& 15& 17& 0 \\
	1& 0.3& 15& 17& 0 \\
	1& 0.4& 15& 17& 0 \\
	1& 0.5& 15& 17& 0 \\
	1& 0.6& 15& 18& 1 \\
	1& 0.7& 15& 19& 2 \\
	1& 0.8& 15& 17& 0 \\
	1& 0.9& 15& 17& 0 \\
	1& 1& 15& 18& 1 \\
	
\end{longtable}

На основе проведённых экспериментов можно сделать следующие выводы:
\begin{itemize}
	\item В наборах параметров, где $\alpha$ и $\rho$ равны 1 и 0 (или 0 и 1), в большинстве случаев, результаты отличаются от эталонного значения.
	\item Наибольше совпадений с эталонным значением оказалось в эксперименте №3, где половина связей между вершинами графа отсутствовала, поскольку диапазон возможных значений значительно сужен.
	\item В эксперименте №1, работающем с матрицей с величинами от 1 до 10, несовпадений гораздо больше, чем в эксперименте №2, где значения варьировались от 1 до 1000. Такая ситуация возникла из-за того, что во втором случае длины маршрутов сильно отличаются ввиду большого диапазона значений, и выделить выигрышные пути проще.
	\item Наибольшее несовпадание составляет примерно 10.64\% от эталонного значения.
\end{itemize}

\section{Измерения}
\qquadДля проведения замеров процессорного времени использовались матрицы размера $N \times N$, где $N \in \left\lbrace 2, 4, 8, 16, 32, 64, 128, 256, 512 \right\rbrace$.
Их содержимое генерируется случайным образом.\\ 

Каждый замер проводится 5 раз для получения более точного среднего результата. \\

В таблице \hyperref[table_time]{таблице 4.4} представлены результаты замеров.

\begin{table}[ph] \label{table_time}
	\caption{Результаты измерений процессорного времени}
	\centering
	\begin{tabular}{|c|c|c|c|c|c|c|c|c|c|}
		\hline
		$N$ &2 &4 &8 &16 &32 &64 &128 &256 &512\\
		\hline
		Время    &4.609e-5 &2.136e-4 &0.00117 &0.007 &0.0458 &0.308 &2.64 &16.525 &123.008\\
		\hline
	\end{tabular}
\end{table}

Можно заметить, что при каждом увеличении $N$ в 2 раза, время, в свою очередь, в среднем, увеличивается примерно в 6.3 раз. Отсюда можно сделать вывод о том, что трудоёмкость ближе к $O(n^3)$.

\section*{Вывод}
\addcontentsline{toc}{section}{Вывод}
\qquadПроведены серии экспериментов для того, чтобы исследовать параметризацию на базе муравьиного алгоритма, замеры процессорного времени, и на основе полученных данных были составлены сравнительные таблицы. В результате анализа получившихся таблиц были сделаны выводы, приведённые выше.
