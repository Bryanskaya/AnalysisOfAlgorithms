%Проведём замеры процессорного времени, которое затрачивается каждым алгоритмом на поиск ключа, найдём масимальное, минимальное и среднее и сравним полученные результаты.


\section{Характеристики ПК}
\qquadПри проведении замеров времени использовался компьютер, имеющий следующие характеристики:
\begin{itemize}
	\item OC - Windows 10 Pro
	\item Процессор - Inter Core i7 10510U (1800 МГц)
	\item Объём ОЗУ - 16 Гб
\end{itemize}

\section{Измерения}
\qquadДля проведения замеров процессорного времени использовались матрицы размера $N \times N$, где $N \in \left\lbrace 2, 4, 8, 16, 32, 64, 128, 256 \right\rbrace$.
Их содержимое генерируется случайным образом.\\ 

Каждый замер проводится 5 раз для получения более точного среднего результата. \\

В таблице \hyperref[table_time]{таблице 4.1} представлены результаты.!!!!!!!!!!!

\begin{table}[ph] \label{table_time}
	\caption{Результаты измерений процессорного времени}
	\centering
	\begin{tabular}{|c|c|c|c|c|c|c|c|c|}
		\hline
		$N$ &2 &4 &8 &16 &32 &64 &128 &256\\
		\hline
		Время    &4.609e-5 &2.136e-4 &0.00117 &0.007 &0.0458 &0.308 &2.64 &16.525\\
		\hline
	\end{tabular}
\end{table}

Можно заметить, что при каждом увеличении $N$ в 2 раза, время, в свою очередь, увеличивается примерно от 4.7 до 6.26 раз. ??????????