Проведём серию экспериментов для исследования параметризации для метода муравьиного алгоритма и сделаем замеры процессорного времени, которое затрачивается им на решение задачи коммивояжёра.

\section{Характеристики ПК}
\qquadПри проведении замеров времени использовался компьютер, имеющий следующие характеристики:
\begin{itemize}
	\item OC - Windows 10 Pro
	\item Процессор - Inter Core i7 10510U (1800 МГц)
	\item Объём ОЗУ - 16 Гб
\end{itemize}

\section{Содержание экспериментов}
\qquadБудут рассмотрены следующие случаи:
\begin{enumerate}
	\item матрица стоимостей размера $8 \times 8$, значения от 1 до 10;
	\item матрица стоимостей размера $8 \times 8$, значения от 1 до 1000;
	\item матрица стоимостей размера $8 \times 8$, значения от 1 до 10, примерно половина элементов равна 0;
\end{enumerate}

Длина получаемого маршрута представляет собой лучшее значение из найденных за 3 прогона алгоритма.

\section{Эксперимент 1}
\qquadБыла использована матрица \ref{matrix_1}.

\begin{scriptsize} 
	\begin{equation}\label{matrix_1}
		\left(
		\begin{array}{cccccccc}
			0& 7& 6& 3& 8& 6& 9& 8\\
			7& 0& 6& 2& 2& 2& 6& 9\\
			6& 6& 0& 6& 4& 7& 8& 8\\
			3& 2& 6& 0& 1& 1& 2& 4\\
			8& 2& 4& 1& 0& 2& 8& 5\\
			6& 2& 7& 1& 2& 0& 1& 2\\
			9& 6& 8& 2& 8& 1& 0& 4\\
			8& 9& 8& 4& 5& 2& 4& 0\\
		\end{array}\right)
\end{equation} \end{scriptsize}

Метод полного перебора определил кратчайший маршрут и его эталонную длину, равную 25. По результатам проведённой параметризации составлена \hyperref[table_1]{таблица 4.1}.


\begin{longtable}{|c|c|c|c|c|}
	\caption{Результаты эксперимента №1}\label{table_1}\\
	\hline
	
	$\alpha$ & $\rho$ & Количество  & Длина & Разница с  \\
	&& поколений & маршрута & эталонным значением\\
	\hline
	\endfirsthead
	
	\hline
	$\alpha$ & $\rho$ & Количество  & Длина & Разница с  \\
	&& поколений & маршрута & эталонным значением\\
	\hline
	\endhead
	
	\hline
	\multicolumn{5}{c}{\textit{Продолжение на следующей странице}}
	\endfoot
	\hline
	\endlastfoot
	
	0& 0& 30& 26& 1\\
	0& 0.1& 30& 25& 0\\
	0& 0.2& 30& 25& 0\\
	0& 0.3& 30& 26& 1\\
	0& 0.4& 30& 27& 2\\
	0& 0.5& 30& 25& 0\\
	0& 0.6& 30& 26& 1\\
	0& 0.7& 30& 25& 0\\
	0& 0.8& 30& 26& 1\\
	0& 0.9& 30& 25& 0\\
	0& 1& 30& 25& 0 \\
	\hline
	
	0.1& 0& 30& 25& 0\\
	0.1& 0.1& 30& 25& 0\\
	0.1& 0.2& 30& 25& 0\\
	0.1& 0.3& 30& 25& 0\\
	0.1& 0.4& 30& 25& 0\\
	0.1& 0.5& 30& 25& 0\\
	0.1& 0.6& 30& 25& 0\\
	0.1& 0.7& 30& 26& 1\\
	0.1& 0.8& 30& 25& 0\\
	0.1& 0.9& 30& 27& 2\\
	0.1& 1& 30& 25& 0\\
	\hline
	
	0.2& 0& 30& 25& 0\\
	0.2& 0.1& 30& 27& 2\\
	0.2& 0.2& 30& 27& 2\\
	0.2& 0.3& 30& 25& 0\\
	0.2& 0.4& 30& 25& 0\\
	0.2& 0.5& 30& 25& 0\\
	0.2& 0.6& 30& 25& 0\\
	0.2& 0.7& 30& 25& 0\\
	0.2& 0.8& 30& 26& 1\\
	0.2& 0.9& 30& 25& 0\\
	0.2& 1& 30& 26& 1\\
	\hline
	
	0.3& 0& 30& 26& 1\\
	0.3& 0.1& 30& 25& 0\\
	0.3& 0.2& 30& 25& 0\\
	0.3& 0.3& 30& 25& 0\\
	0.3& 0.4& 30& 25& 0\\
	0.3& 0.5& 30& 25& 0\\
	0.3& 0.6& 30& 25& 0\\
	0.3& 0.7& 30& 27& 2\\
	0.3& 0.8& 30& 25& 0\\
	0.3& 0.9& 30& 25& 0\\
	0.3& 1& 30& 26& 1\\
	\hline
	
	0.4& 0& 30& 25& 0\\
	0.4& 0.1& 30& 25& 0\\
	0.4& 0.2& 30& 25& 0\\
	0.4& 0.3& 30& 25& 0\\
	0.4& 0.4& 30& 25& 0\\
	0.4& 0.5& 30& 25& 0\\
	0.4& 0.6& 30& 25& 0\\
	0.4& 0.7& 30& 25& 0\\
	0.4& 0.8& 30& 26& 1\\
	0.4& 0.9& 30& 26& 1\\
	0.4& 1& 30& 25& 0\\
	\hline
	
	0.5& 0& 30& 25& 0\\
	0.5& 0.1& 30& 26& 1\\
	0.5& 0.2& 30& 26& 1\\
	0.5& 0.3& 30& 26& 1\\
	0.5& 0.4& 30& 26& 1\\
	0.5& 0.5& 30& 25& 0\\
	0.5& 0.6& 30& 25& 0\\
	0.5& 0.7& 30& 25& 0\\
	0.5& 0.8& 30& 27& 2\\
	0.5& 0.9& 30& 25& 0\\
	0.5& 1& 30& 25& 0\\
	\hline
	
	0.6& 0& 30& 25& 0\\
	0.6& 0.1& 30& 25& 0\\
	0.6& 0.2& 30& 25& 0\\
	0.6& 0.3& 30& 25& 0\\
	0.6& 0.4& 30& 25& 0\\
	0.6& 0.5& 30& 25& 0\\
	0.6& 0.6& 30& 25& 0\\
	0.6& 0.7& 30& 25& 0\\
	0.6& 0.8& 30& 25& 0\\
	0.6& 0.9& 30& 25& 0\\
	0.6& 1& 30& 25& 0\\
	\hline
	
	0.7& 0& 30& 25& 0\\
	0.7& 0.1& 30& 25& 0\\
	0.7& 0.2& 30& 25& 0\\
	0.7& 0.3& 30& 25& 0\\
	0.7& 0.4& 30& 25& 0\\
	0.7& 0.5& 30& 25& 0\\
	0.7& 0.6& 30& 25& 0\\
	0.7& 0.7& 30& 27& 2\\
	0.7& 0.8& 30& 25& 0\\
	0.7& 0.9& 30& 25& 0\\
	0.7& 1& 30& 25& 0\\
	\hline
	
	0.8& 0& 30& 25& 0\\
	0.8& 0.1& 30& 25& 0\\
	0.8& 0.2& 30& 25& 0\\
	0.8& 0.3& 30& 25& 0\\
	0.8& 0.4& 30& 25& 0\\
	0.8& 0.5& 30& 25& 0\\
	0.8& 0.6& 30& 25& 0\\
	0.8& 0.7& 30& 25& 0\\
	0.8& 0.8& 30& 25& 0\\
	0.8& 0.9& 30& 25& 0\\
	0.8& 1& 30& 25& 0\\
	\hline
	
	0.9& 0& 30& 25& 0\\
	0.9& 0.1& 30& 25& 0\\
	0.9& 0.2& 30& 25& 0\\
	0.9& 0.3& 30& 26& 1\\
	0.9& 0.4& 30& 26& 1\\
	0.9& 0.5& 30& 25& 0\\
	0.9& 0.6& 30& 25& 0\\
	0.9& 0.7& 30& 25& 0\\
	0.9& 0.8& 30& 25& 0\\
	0.9& 0.9& 30& 25& 0\\
	0.9& 1& 30& 25& 0\\
	\hline
	
	1& 0& 30& 25& 0\\
	1& 0.1& 30& 25& 0\\
	1& 0.2& 30& 25& 0\\
	1& 0.3& 30& 25& 0\\
	1& 0.4& 30& 25& 0\\
	1& 0.5& 30& 25& 0\\
	1& 0.6& 30& 27& 2\\
	1& 0.7& 30& 25& 0\\
	1& 0.8& 30& 25& 0\\
	1& 0.9& 30& 25& 0\\
	1& 1& 30& 25& 0\\
	
\end{longtable}


\section{Эксперимент 2}
\qquadБыла использована матрица \ref{matrix_2}.

\begin{scriptsize} 
	\begin{equation}\label{matrix_2}
		\left(
		\begin{array}{cccccccc}
			 0& 233& 767& 803& 647& 749& 59& 295\\
			233&  0& 44& 894& 886& 370& 283& 157\\
			767& 44&  0& 711& 370& 468& 22& 370\\
			803& 894& 711&  0& 329& 36& 243& 948\\
			647& 886& 370& 329&  0& 525& 241& 117\\
			749& 370& 468& 36& 525&  0& 370& 834\\
			59& 283& 22& 243& 241& 370&  0& 201\\
			295& 157& 370& 948& 117& 834& 201&  0\\
		\end{array}\right)
\end{equation} \end{scriptsize}

Метод полного перебора определил кратчайший маршрут и его эталонную длину, равную 1272. По результатам проведённой параметризации составлена \hyperref[table_2]{таблица 4.2}.


\begin{longtable}{|c|c|c|c|c|}
	\caption{Результаты эксперимента №2}\label{table_2}\\
	\hline
	
	$\alpha$ & $\rho$ & Количество  & Длина & Разница с  \\
	&& поколений & маршрута & эталонным значением\\
	\hline
	\endfirsthead
	
	\hline
	$\alpha$ & $\rho$ & Количество  & Длина & Разница с  \\
	&& поколений & маршрута & эталонным значением\\
	\hline
	\endhead
	
	\hline
	\multicolumn{5}{c}{\textit{Продолжение на следующей странице}}
	\endfoot
	\hline
	\endlastfoot
	
	0& 0& 30& 1272& 0\\
	0& 0.1& 30& 1272& 0\\
	0& 0.2& 30& 1272& 0\\
	0& 0.3& 30& 1272& 0\\
	0& 0.4& 30& 1272& 0\\
	0& 0.5& 30& 1272& 0\\
	0& 0.6& 30& 1272& 0\\
	0& 0.7& 30& 1272& 0\\
	0& 0.8& 30& 1272& 0\\
	0& 0.9& 30& 1272& 0\\
	0& 1& 30& 1272& 0\\
	\hline
	
	0.1& 0& 30& 1272& 0\\
	0.1& 0.1& 30& 1272& 0\\
	0.1& 0.2& 30& 1272& 0\\
	0.1& 0.3& 30& 1272& 0\\
	0.1& 0.4& 30& 1272& 0\\
	0.1& 0.5& 30& 1272& 0\\
	0.1& 0.6& 30& 1272& 0\\
	0.1& 0.7& 30& 1272& 0\\
	0.1& 0.8& 30& 1272& 0\\
	0.1& 0.9& 30& 1272& 0\\
	0.1& 1& 30& 1272& 0\\
	\hline
	
	0.2& 0& 30& 1272& 0\\
	0.2& 0.1& 30& 1272& 0\\
	0.2& 0.2& 30& 1272& 0\\
	0.2& 0.3& 30& 1272& 0\\
	0.2& 0.4& 30& 1272& 0\\
	0.2& 0.5& 30& 1272& 0\\
	0.2& 0.6& 30& 1272& 0\\
	0.2& 0.7& 30& 1272& 0\\
	0.2& 0.8& 30& 1272& 0\\
	0.2& 0.9& 30& 1272& 0\\
	0.2& 1& 30& 1272& 0\\
	\hline
	
	0.3& 0& 30& 1272& 0\\
	0.3& 0.1& 30& 1272& 0\\
	0.3& 0.2& 30& 1272& 0\\
	0.3& 0.3& 30& 1272& 0\\
	0.3& 0.4& 30& 1272& 0\\
	0.3& 0.5& 30& 1272& 0\\
	0.3& 0.6& 30& 1272& 0\\
	0.3& 0.7& 30& 1272& 0\\
	0.3& 0.8& 30& 1272& 0\\
	0.3& 0.9& 30& 1272& 0\\
	0.3& 1& 30& 1272& 0\\
	\hline
	
	0.4& 0& 30& 1272& 0\\
	0.4& 0.1& 30& 1272& 0\\
	0.4& 0.2& 30& 1272& 0\\
	0.4& 0.3& 30& 1272& 0\\
	0.4& 0.4& 30& 1272& 0\\
	0.4& 0.5& 30& 1272& 0\\
	0.4& 0.6& 30& 1272& 0\\
	0.4& 0.7& 30& 1272& 0\\
	0.4& 0.8& 30& 1272& 0\\
	0.4& 0.9& 30& 1272& 0\\
	0.4& 1& 30& 1272& 0\\
	\hline
	
	0.5& 0& 30& 1272& 0\\
	0.5& 0.1& 30& 1272& 0\\
	0.5& 0.2& 30& 1272& 0\\
	0.5& 0.3& 30& 1272& 0\\
	0.5& 0.4& 30& 1272& 0\\
	0.5& 0.5& 30& 1272& 0\\
	0.5& 0.6& 30& 1272& 0\\
	0.5& 0.7& 30& 1272& 0\\
	0.5& 0.8& 30& 1272& 0\\
	0.5& 0.9& 30& 1272& 0\\
	0.5& 1& 30& 1272& 0\\
	\hline
	
	0.6& 0& 30& 1272& 0\\
	0.6& 0.1& 30& 1272& 0\\
	0.6& 0.2& 30& 1272& 0\\
	0.6& 0.3& 30& 1272& 0\\
	0.6& 0.4& 30& 1272& 0\\
	0.6& 0.5& 30& 1272& 0\\
	0.6& 0.6& 30& 1272& 0\\
	0.6& 0.7& 30& 1272& 0\\
	0.6& 0.8& 30& 1272& 0\\
	0.6& 0.9& 30& 1272& 0\\
	0.6& 1& 30& 1272& 0\\
	\hline
	
	0.7& 0& 30& 1272& 0\\
	0.7& 0.1& 30& 1272& 0\\
	0.7& 0.2& 30& 1272& 0\\
	0.7& 0.3& 30& 1272& 0\\
	0.7& 0.4& 30& 1272& 0\\
	0.7& 0.5& 30& 1272& 0\\
	0.7& 0.6& 30& 1272& 0\\
	0.7& 0.7& 30& 1272& 0\\
	0.7& 0.8& 30& 1272& 0\\
	0.7& 0.9& 30& 1272& 0\\
	0.7& 1& 30& 1272& 0\\
	\hline
	
	0.8& 0& 30& 1272& 0\\
	0.8& 0.1& 30& 1272& 0\\
	0.8& 0.2& 30& 1272& 0\\
	0.8& 0.3& 30& 1272& 0\\
	0.8& 0.4& 30& 1272& 0\\
	0.8& 0.5& 30& 1272& 0\\
	0.8& 0.6& 30& 1272& 0\\
	0.8& 0.7& 30& 1272& 0\\
	0.8& 0.8& 30& 1272& 0\\
	0.8& 0.9& 30& 1272& 0\\
	0.8& 1& 30& 1272& 0\\
	\hline
	
	0.9& 0& 30& 1272& 0\\
	0.9& 0.1& 30& 1272& 0\\
	0.9& 0.2& 30& 1272& 0\\
	0.9& 0.3& 30& 1272& 0\\
	0.9& 0.4& 30& 1272& 0\\
	0.9& 0.5& 30& 1272& 0\\
	0.9& 0.6& 30& 1272& 0\\
	0.9& 0.7& 30& 1272& 0\\
	0.9& 0.8& 30& 1272& 0\\
	0.9& 0.9& 30& 1272& 0\\
	0.9& 1& 30& 1272& 0\\
	\hline
	
	1& 0& 30& 1272& 0\\
	1& 0.1& 30& 1272& 0\\
	1& 0.2& 30& 1272& 0\\
	1& 0.3& 30& 1272& 0\\
	1& 0.4& 30& 1272& 0\\
	1& 0.5& 30& 1272& 0\\
	1& 0.6& 30& 1272& 0\\
	1& 0.7& 30& 1272& 0\\
	1& 0.8& 30& 1421& 149\\
	1& 0.9& 30& 1272& 0\\
	1& 1& 30& 1421& 149\\

\end{longtable}

\section{Эксперимент 3}
\qquadБыла использована матрица \ref{matrix_3}.

\begin{scriptsize} 
	\begin{equation}\label{matrix_3}
		\left(
		\begin{array}{cccccccc}
			0& 8& 4& 0& 0& 7& 0& 0\\
			8&  0& 0& 1& 0& 0& 9& 8\\
			4& 0&  0& 1& 1& 1& 0& 6\\
			0& 1& 1&  0& 0& 2& 5& 0\\
			0& 0& 1& 0&  0& 0& 8& 3\\
			7& 0& 1& 2& 0&  0& 5& 0\\
			0& 9& 0& 5& 8& 5&  0& 4\\
			0& 8& 6& 0& 3& 0& 4&  0\\
		\end{array}\right)
\end{equation} \end{scriptsize}

Метод полного перебора определил кратчайший маршрут и его эталонную длину, равную 28. По результатам проведённой параметризации составлена \hyperref[table_3]{таблица 4.3}.


\begin{longtable}{|c|c|c|c|c|}
	\caption{Результаты эксперимента №3}\label{table_3}\\
	\hline
	
	$\alpha$ & $\rho$ & Количество  & Длина & Разница с  \\
	&& поколений & маршрута & эталонным значением\\
	\hline
	\endfirsthead
	
	\hline
	$\alpha$ & $\rho$ & Количество  & Длина & Разница с  \\
	&& поколений & маршрута & эталонным значением\\
	\hline
	\endhead
	
	\hline
	\multicolumn{5}{c}{\textit{Продолжение на следующей странице}}
	\endfoot
	\hline
	\endlastfoot
	
	0& 0& 30& 28& 0\\
	0& 0.1& 30& 28& 0\\
	0& 0.2& 30& 28& 0\\
	0& 0.3& 30& 28& 0\\
	0& 0.4& 30& 28& 0\\
	0& 0.5& 30& 28& 0\\
	0& 0.6& 30& 28& 0\\
	0& 0.7& 30& 28& 0\\
	0& 0.8& 30& 28& 0\\
	0& 0.9& 30& 28& 0\\
	0& 1& 30& 28& 0\\
	\hline
	
	0.1& 0& 30& 28& 0\\
	0.1& 0.1& 30& 28& 0\\
	0.1& 0.2& 30& 28& 0\\
	0.1& 0.3& 30& 28& 0\\
	0.1& 0.4& 30& 28& 0\\
	0.1& 0.5& 30& 28& 0\\
	0.1& 0.6& 30& 28& 0\\
	0.1& 0.7& 30& 28& 0\\
	0.1& 0.8& 30& 28& 0\\
	0.1& 0.9& 30& 28& 0\\
	0.1& 1& 30& 28& 0\\
	\hline
	
	0.2& 0& 30& 28& 0\\
	0.2& 0.1& 30& 28& 0\\
	0.2& 0.2& 30& 28& 0\\
	0.2& 0.3& 30& 28& 0\\
	0.2& 0.4& 30& 28& 0\\
	0.2& 0.5& 30& 28& 0\\
	0.2& 0.6& 30& 28& 0\\
	0.2& 0.7& 30& 28& 0\\
	0.2& 0.8& 30& 28& 0\\
	0.2& 0.9& 30& 28& 0\\
	0.2& 1& 30& 28& 0\\
	\hline
	
	0.3& 0& 30& 28& 0\\
	0.3& 0.1& 30& 28& 0\\
	0.3& 0.2& 30& 28& 0\\
	0.3& 0.3& 30& 28& 0\\
	0.3& 0.4& 30& 28& 0\\
	0.3& 0.5& 30& 28& 0\\
	0.3& 0.6& 30& 28& 0\\
	0.3& 0.7& 30& 28& 0\\
	0.3& 0.8& 30& 28& 0\\
	0.3& 0.9& 30& 28& 0\\
	0.3& 1& 30& 28& 0\\
	\hline
	
	0.4& 0& 30& 28& 0\\
	0.4& 0.1& 30& 28& 0\\
	0.4& 0.2& 30& 28& 0\\
	0.4& 0.3& 30& 28& 0\\
	0.4& 0.4& 30& 28& 0\\
	0.4& 0.5& 30& 28& 0\\
	0.4& 0.6& 30& 28& 0\\
	0.4& 0.7& 30& 28& 0\\
	0.4& 0.8& 30& 28& 0\\
	0.4& 0.9& 30& 28& 0\\
	0.4& 1& 30& 28& 0\\
	\hline
	
	0.5& 0& 30& 28& 0\\
	0.5& 0.1& 30& 28& 0\\
	0.5& 0.2& 30& 28& 0\\
	0.5& 0.3& 30& 28& 0\\
	0.5& 0.4& 30& 28& 0\\
	0.5& 0.5& 30& 28& 0\\
	0.5& 0.6& 30& 28& 0\\
	0.5& 0.7& 30& 28& 0\\
	0.5& 0.8& 30& 28& 0\\
	0.5& 0.9& 30& 28& 0\\
	0.5& 1& 30& 28& 0\\
	\hline
	
	0.6& 0& 30& 28& 0\\
	0.6& 0.1& 30& 28& 0\\
	0.6& 0.2& 30& 28& 0\\
	0.6& 0.3& 30& 28& 0\\
	0.6& 0.4& 30& 28& 0\\
	0.6& 0.5& 30& 28& 0\\
	0.6& 0.6& 30& 28& 0\\
	0.6& 0.7& 30& 28& 0\\
	0.6& 0.8& 30& 28& 0\\
	0.6& 0.9& 30& 28& 0\\
	0.6& 1& 30& 28& 0\\
	\hline
	
	0.7& 0& 30& 28& 0\\
	0.7& 0.1& 30& 28& 0\\
	0.7& 0.2& 30& 28& 0\\
	0.7& 0.3& 30& 28& 0\\
	0.7& 0.4& 30& 28& 0\\
	0.7& 0.5& 30& 28& 0\\
	0.7& 0.6& 30& 28& 0\\
	0.7& 0.7& 30& 28& 0\\
	0.7& 0.8& 30& 28& 0\\
	0.7& 0.9& 30& 28& 0\\
	0.7& 1& 30& 28& 0\\
	\hline
	
	0.8& 0& 30& 28& 0\\
	0.8& 0.1& 30& 28& 0\\
	0.8& 0.2& 30& 28& 0\\
	0.8& 0.3& 30& 28& 0\\
	0.8& 0.4& 30& 28& 0\\
	0.8& 0.5& 30& 28& 0\\
	0.8& 0.6& 30& 28& 0\\
	0.8& 0.7& 30& 28& 0\\
	0.8& 0.8& 30& 28& 0\\
	0.8& 0.9& 30& 28& 0\\
	0.8& 1& 30& 28& 0\\
	\hline
	
	0.9& 0& 30& 28& 0\\
	0.9& 0.1& 30& 28& 0\\
	0.9& 0.2& 30& 28& 0\\
	0.9& 0.3& 30& 28& 0\\
	0.9& 0.4& 30& 28& 0\\
	0.9& 0.5& 30& 28& 0\\
	0.9& 0.6& 30& 28& 0\\
	0.9& 0.7& 30& 28& 0\\
	0.9& 0.8& 30& 28& 0\\
	0.9& 0.9& 30& 28& 0\\
	0.9& 1& 30& 28& 0\\
	\hline
	
	1& 0& 30& 28& 0\\
	1& 0.1& 30& 28& 0\\
	1& 0.2& 30& 28& 0\\
	1& 0.3& 30& 28& 0\\
	1& 0.4& 30& 28& 0\\
	1& 0.5& 30& 28& 0\\
	1& 0.6& 30& 28& 0\\
	1& 0.7& 30& 28& 0\\
	1& 0.8& 30& 28& 0\\
	1& 0.9& 30& 28& 0\\
	1& 1& 30& 28& 0\\
	
\end{longtable}

На основе проведённых экспериментов можно сделать следующие выводы:
\begin{itemize}
	\item В наборах параметров, где $\alpha$ и $\rho$ равны 1 и 0 (или 0 и 1), в большинстве случаев, результаты отличаются от эталонного значения.
	\item Наибольше совпадений с эталонным значением оказалось в эксперименте №3, где половина связей между вершинами графа отсутствовала, поскольку диапазон возможных значений значительно сужен.
	\item В эксперименте №1, работающем с матрицей с величинами от 1 до 10, несовпадений гораздо больше, чем в эксперименте №2, где значения варьировались от 1 до 1000. Такая ситуация возникла из-за того, что во втором случае длины маршрутов сильно отличаются ввиду большого диапазона значений, и выделить выигрышные пути проще.
	\item Наибольшее несовпадание составляет примерно 11.71\% от эталонного значения.
\end{itemize}

\section{Измерения}
\qquadДля проведения замеров процессорного времени использовались матрицы размера $N \times N$, где $N \in \left\lbrace 2, 4, 8, 16, 32, 64, 128, 256б 512 \right\rbrace$.
Их содержимое генерируется случайным образом.\\ 

Каждый замер проводится 5 раз для получения более точного среднего результата. \\

В таблице \hyperref[table_time]{таблице 4.4} представлены результаты замеров.

\begin{table}[ph] \label{table_time}
	\caption{Результаты измерений процессорного времени}
	\centering
	\begin{tabular}{|c|c|c|c|c|c|c|c|c|c|}
		\hline
		$N$ &2 &4 &8 &16 &32 &64 &128 &256 &512\\
		\hline
		Время    &4.609e-5 &2.136e-4 &0.00117 &0.007 &0.0458 &0.308 &2.64 &16.525 &123.008\\
		\hline
	\end{tabular}
\end{table}

Можно заметить, что при каждом увеличении $N$ в 2 раза, время, в свою очередь, в среднем, увеличивается примерно в 6.3 раз. Отсюда можно сделать вывод о том, что трудоёмкость ближе к $O(n^3)$.

\section*{Вывод}
\addcontentsline{toc}{section}{Вывод}
\qquadПроведены серии экспериментов для того, чтобы исследовать параметризацию на базе муравьиного алгоритма, замеры процессорного времени, и на основе полученных данных были составлены сравнительные таблицы. В результате анализа получившихся таблиц были сделаны выводы, приведённые выше.
