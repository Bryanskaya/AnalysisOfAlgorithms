В ходе лабораторной работы была достигнута поставленная цель, а именно, проведён сравнительный анализ метода полного перебора и муравьиного алгоритма. \\

В процессе выполнения были решены все задачи. Описаны решаемая задача коммивояжёра, все рассматриваемые алгоритмы. Все проработанные алгоритмы реализованы, кроме того, были проведены замеры процессорного времени работы, оценена трудоёмкость муравьиного алгоритма и проведена его параметризация. На основе полученных данных сделан сравнительный анализ, сформулированы выводы.\\

Алгоритм полного перебора всегда находит корректное решение, но в силу того, что просматривает все возможные перестановки вершин, является очень трудоёмким по времени, поэтому может применяться на ограниченном множестве задач.\\

По результатам экспериментов были сделаны следующие заключения.
\begin{itemize}
	\item Если параметры $\alpha$ и $\rho$ принимают граничные значения, то результат в большинстве случае отличается от эталонного, который был найден полным перебором.
	\item Чем в большем диапазоне значений лежат элементы матрицы стоимостей, тем больше вероятность того, что результат работы муравьиного алгоритма будет максимально приближен к эталонному (и скорее всего, даже совпадать), в силу меньшего числа потенциальных решений по сравнению с ситуацией, когда элементы принадлежат сравнительно меньшему числовому отрезку.
	\item В случае, когда значительная часть связей между вершинами отсутствует, муравьиный алгоритм демонстрирует больше совпадений, чем в случае плотно заполненной матрицы, ввиду сокращения количества возможных вариантов маршрутов.
\end{itemize}

Трудоёмкость алгоритма полного перебора -- $O(n!)$. Что касается муравьиного алгоритма, то по результатам замеров процессорного времени было выявлено, что его трудоёмкость стремится к $O(n^3)$. Поэтому диапазон применения муравьиного алгоритма заметно больше, чем полного перебора.

