В этом разделе будут поставлены цель и основные задачи лабораторной работы, которые будут решаться по мере её выполнения.

\section{Цель и задачи}
\qquad\textbf{Цель} данной работы: провести сравнительный анализ метода полного перебора и эвристического метода на базе муравьиного алгоритма.\\

Для достижения поставленной цели необходимо решить следующий ряд \textbf{задач}:
\begin{enumerate}
	\item[1)] дать описание базовой задачи;
	\item[2)] описать алгоритмы;
	\item[3)] реализовать все рассмотренные алгоритмы;
	\item[4)] провести парметризацию муравьиного алгоритма для выбранного класса задач, то есть определить такие комбинации параметров или их диапазонов, при которых метод даёт наилучшие результаты.
\end{enumerate}

\section{Задача о коммивояжёре}
\qquadВ задаче о коммивояжёре, которая тесно связана с задачей о гамильтоновом цикле, коммивояжёр должен посетить $n$ городов. Можно сказать, что коммивояжёру нужно совершить тур, или гамильтонов цикл, посетив каждый город ровно по одному разу и, завершив путешествие в том же городе, из которого он выехал. С каждым переездом из города $i$ в город $j$ связана некоторая стоимость пути $c(i, j)$, выраженная целым неотрицательным числом, и коммивояжёру нужно совершить тур таким образом, чтобы общая стоимость (т.е. сумма стоимостей всех переездов) была минимальной. \cite{cite_travsale}\\

Для решения этой задачи предлагается два следующих алгоритма.

\section{Алгоритм полного перебора}
\qquadЭтот алгоритм заключается в полном переборе всех возможных комбинаций точек (городов). На вход подаётся число городов $N$ и матрица стоимостей C. Так как количество городов равно $N$, то последовательно будут рассматриваться все перестановки из $N - 1$ положительных целых чисел. Будет анализироваться каждый из этих возможных туров, и будет выбран тот, у которого наименьшая стоимость. \cite{Kormen}

Этот алгоритм достаточно точный, но продолжительность таких вычислений может занять непозволительно много времени.\\

\section{Муравьиный алгоритм}
\qquadВ его основе лежит моделирование колонии муравьёв. Сама по себе колония представляет собой систему, в которой строго определены правила автономного поведения особей.