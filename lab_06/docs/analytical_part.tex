В этом разделе будут поставлены цель и основные задачи лабораторной работы, которые будут решаться по мере её выполнения.

\section{Цель и задачи}
\qquad\textbf{Цель} данной работы: провести сравнительный анализ метода полного перебора и эвристического метода на базе муравьиного алгоритма.\\

Для достижения поставленной цели необходимо решить следующий ряд \textbf{задач}:
\begin{enumerate}
	\item[1)] дать описание базовой задачи;
	\item[2)] описать алгоритмы;
	\item[3)] реализовать все рассмотренные алгоритмы;
	\item[4)] провести парметризацию муравьиного алгоритма для выбранного класса задач, то есть определить такие комбинации параметров или их диапазонов, при которых метод даёт наилучшие результаты.
\end{enumerate}

\section{Задача о коммивояжёре}
\qquadВ задаче о коммивояжёре, которая тесно связана с задачей о гамильтоновом цикле, коммивояжёр должен посетить $n$ городов. Можно сказать, что коммивояжёру нужно совершить тур, или гамильтонов цикл, посетив каждый город ровно по одному разу и завершив путешествие в том же городе, из которого он выехал. С каждым переездом из города $i$ в город $j$ связана некоторая стоимость пути $c(i, j)$, выраженная целым неотрицательным числом, и коммивояжёру нужно совершить тур таким образом, чтобы общая стоимость (т.е. сумма стоимостей всех переездов) была минимальной. \cite{cite_travsale}\\

Для решения этой задачи предлагается два следующих алгоритма.

\section{Алгоритм полного перебора}
\qquadЭтот алгоритм заключается в полном переборе всех возможных комбинаций точек (городов). На вход подаётся число городов $N$ и матрица стоимостей C. Так как количество городов равно $N$, то последовательно будут рассматриваться все перестановки из $N - 1$ положительных целых чисел. Будет анализироваться каждый из этих возможных туров, и будет выбран тот, у которого наименьшая стоимость. \cite{Kormen}

Этот алгоритм всегда верно находит решение, но продолжительность таких вычислений может занять много времени.\\

\section{Муравьиный алгоритм}
\qquadВ его основе лежит моделирование колонии муравьёв, которая существует $t$ дней. Колония представляет собой систему, в которой строго определены правила автономного поведения особей. Особенностью является то, что члены колонии могут быстро находить кратчайший путь и обмениваться этой информацией. В начале каждого $t$-ого дня осуществляется проход очередной партии муравьёв по городам. \\

В природе происходит непрямой обмен информацией, при котором одна особь изменяет некоторую область окружающей среды, а другие используют эту информацию, когда попадают в эту область. Каждый муравей оставляют особое вещество -- феромон, по которому ориентируются следующие за ним особи. Чем больше муравьёв проходит по какому-либо участку, тем выше на нём концентрация этого вещества, тем самым насекомые помечают наиболее выгодные пути. \\

В алгоритме происходит моделирование такого поведения на некотором графе, ребра которого представляют собой возможные пути перемещения, и, как результат, наиболее обогащённый феромонами путь по рёбрам этого графа и будет решением поставленной задачи.\\

В основе муравьиного алгоритма лежат следующие принципы.
\begin{itemize}
	\item В силу того, что каждый город должен быть посещён только один раз, у каждого муравья в колонии хранится информация об уже посещённых пунктах (далее этот список будет обозначаться как $J$);
	
	\item Муравьи обладают <<зрением>>, то есть желанием посетить тот или иной город $j$, и эта величина рассчитывается по формуле \ref{formula1}.
	\begin{equation}\label{formula1}
			\eta_{ij} = \frac{1}{D_{ij}},
		\end{equation}
	где $D_{ij}$ - расстояние между городами $i$ и $j$.
	
	\item Как говорилось выше, немаловажную роль в выборе следующего участка пути, играет феромон, который оставили другие муравьи, его количество обозначается как $\tau_{ij}$. И вероятность того, что дальнейший маршрут $k$-ого муравья будет построен по текущему ребру определяется по формуле \ref{formula2}.
	\begin{equation}\label{formula2}
		\left\{
			\begin{array}{ccc}
				P_{ij, k}(t) = \frac{[\tau_{ij}(t)]^\alpha \cdot [\eta_{ij} ]^\beta}{ \sum\limits_{l \in J_{i, k}} [\tau_{il}(t)]^\alpha \cdot [\eta_{ij}] ^\beta}, j \in J_{i, k};\\
				P_{ij, k}(t) = 0, j \notin J_{i, k}, \\
			\end{array}
		\right.
	\end{equation}
	где $\alpha, \beta$ -- коэффициенты стадности и жадности, то есть, параметры, которые задают веса феромона, при $\alpha = 0$ алгоритм вырождается до жадного алгоритма (то есть будет выбран ближайший город).
	\end{itemize}

В случае, если муравью удалось построить маршрут, удовлетворяющий всем требованиям, на задействованых рёбрах увеличивается количество феромона. Насколько изменится величина этого вещества на конкретном участке, рассчитывается следующим образом (формула \ref{formula3}).
\begin{equation}\label{formula3}
		\Delta \tau_{ij, k}(t) = \left\{
			\begin{array}{ccc}
				\frac{Q}{L_{k}(t)}, (i, j) \in T_{k}(t); \\
				0, (i, j) \notin T_{k}(t),
			\end{array}
		\right.
\end{equation}
где $Q$ -- параметр, имеющий значение порядка длины оптимального пути, $L_k$ -- длина маршрута, $T_k(t)$ -- маршрут, пройденный $k$-ым муравьём к моменту времени $t$.\\

Помимо всего прочего, алгоритм учитывает, подобно природной экосистеме, испарение феромона с наступлением ночи $t$-ого дня, причём с такой скоростью, чтобы не забывались хорошие решения и не возникало преждевременной сходимости. Рассчёты производятся по формуле \ref{formula4}.
\begin{equation}\label{formula4}
	\tau_{ij}(t + 1) = (1 - \rho)\cdot\tau_{ij}(t) + \Delta \tau_{ij}(t); \; \; \Delta \tau_{ij}(t) = \sum\limits_{k = 1}^m \Delta \tau_{ij, k}(t),
\end{equation}
где $\rho \in [0, 1]$ -- коэффициент испарения, $m$ -- количество муравьёв в колонии.\\

Для большей модификации алгоритма используются <<элитные>> муравьи, которые усиливают рёбра наилучшего маршрута, найденного с начала работы алгоритма. \cite{Ulyanov}\\

В результате работы алгоритма находится кратчайший маршрут среди всех тех, которые были найдены. Но важно обратить внимание на то, что не всегда этот маршрут совпадёт с решением, которое можно получить с помощью алгоритма полного перебора, ввиду определённых причин. Можно получить возможно не совсем точное, но очень близкое к идеальному значение.\\

\section*{Вывод}
\addcontentsline{toc}{section}{Вывод}
\qquadБыли поставлены цель и задачи текущей лабораторной работы, также дано описание рассматриваемых алгоритмов.








