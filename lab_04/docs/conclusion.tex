В ходе лабораторной работы была достигнута поставленная цель, а именно, разработаны и исследованы параллельные алгоритмы умножения матриц методом Винограда.\\

В процессе выполнения были решены все задачи. Описаны все рассматриваемые алгоритмы. Все проработанные алгоритмы реализованы, кроме того, были проведены замеры процессорного времени работы на материале серии экспериментов и проведён сравнительный анализ, сделаны выводы.\\

По результатам замеров процессорного времени сделаны следующие заключения.
\begin{itemize}
	\item Использование параллельных алгоритмов на одном потоке неэффективно по времени, так как присутствуют значительные временные затраты на организацию многопоточности, и результат по времени примерно на 20\% уступает последовательному алгоритмому. 
	\item Но уже с двух потоков многопоточные алгоритмы оказываются эффективнее.
	\item Метод, распараллеливающий алгоритм Винограда по строкам, затрачивает меньше времени на умножение матриц, чем алгоритм, работающий по столбцам, до четырёх потоков разница примерно 3\%, но с восьми принимает значения от 15\% до 30\%.
	\item Самым быстродейственным по времени среди разработанных алгоритмов оказался 1 алгоритм (распараллеливание по строкам). Примерно на 53\% он лучше последовательного алгоритма Винограда на матрицах размером меньше $1000 \times 1000$ и далее его выигрыш растёт ещё больше (на $1000 \times 1000$ он составляет больше 70\%).
	\item Увеличив число рабочих потоков до тридцати двух, не удалось получить выигрыша по времени по сравнению с предыдущими показателями, наоборот, затрачиваемое время увеличилось.
\end{itemize}