В этом разделе будут поставлены цель и основные задачи лабораторной работы, которые будут решаться в ходе её выполнения.

\section{Цель и задачи}
\qquad\textbf{Цель} данной работы: разработка и исследование параллельных алгоритмов умножения матриц методом Винограда.\\

Для достижения поставленной цели необходимо решить ряд следующих \textbf{задач}:
\begin{enumerate}
	\item[1)] ввести понятие параллелизма;
	\item[2)] описать алгоритм;
	\item[3)] реализовать его;
	\item[4)] сделать замеры процессорного времени работы на материале серии экспериментов;
	\item[5)] провести сравнительный анализ многопоточных алгоритмов с базовым (последовательным).
\end{enumerate}

Умножение осуществляется над матрицами $A[M \times N]$ и $B[N \times Q]$. Число столбцов первой матрицы должно совпадать с числом строк второй, а таком случае можно осуществлять умножение. Результатом является матрица $C[M \times Q]$, в которой число строк столько же, сколько в первой, а столбцов, столько же, сколько во второй.

\section{Общие принципы работы в параллельном режиме}
\qquadПод \textbf{потоком} подразумевается непрерывнная часть кода процесса, которая может выполняться с другими частями выполняемой программы.\\

Для решения задачи с задействованием нескольких рабочих потоков нужен главный --- \textbf{диспетчер}. Он создаёт рабочие потоки, передаёт каждому из них \textbf{делегат} (указатель на соответствующую функцию). Далее диспетчер запускает их, в качестве аргументов могут быть переданы слудеющие величины:
\begin{itemize}
	\item границы ответственности;
	\item ссылка на исходные данные;
	\item ссылка на память, в которую необходимо записать ответ;
	\item примитивы синхронизаций.
\end{itemize}

Главный поток создаёт массив сброшенных \textbf{семафоров}, каждый из которых закреплен за определённым потоком. После того, как рабочий поток выполнит поставленную задачу, сохранит или передаст свою часть решения, он устанавливает свой семафор и заканчивает работу.

\section{Алгоритм Винограда}
\qquadОбозначим строку $A_{i,*}$ как $\overrightarrow{u}$, $B_{*,j}$ как $\overrightarrow{v}$.
Пусть $u = (u_1, u_2, u_3, u_4)$ и $v = (v_1, v_2, v_3, v_4)$, тогда их произведение описывается формулой \ref{formula2}.
\begin{equation}\label{formula2}
	u \cdot v = u_1 \cdot v_1 + u_2 \cdot v_2 + u_3 \cdot v_3 + u_4 \cdot v_4
\end{equation}

Выражение (\ref{formula2}) можно преобразовать к виду \ref{formula3}.
\begin{equation}\label{formula3}
	u \cdot v = (u_1 + v_2)\cdot(u_2 + v_1) + (u_3 + v_1)\cdot(u_4 + v_3) - u_1\cdot u_2 - u_3\cdot u_4 - v_1\cdot v_2 - v_3\cdot v_4
\end{equation}

Причём, при нечётном значении N нужно отдельно учесть ещё одно слагаемое $u_5 \cdot v_5$. \\

Алгоритм Винограда основывается на раздельной работе со слагаемыми из выражения (\ref{formula3}).

\section{Параллельный алгоритм Винограда}
\qquadВ силу того, что вычисление результата для каждой строки/стобца не зависит от результатов других строк/столбцов, можно распараллелить эти действия.
 
В алгоритме Винограда формирование конечной матрицы занимает большую часть времени работы всего алгоритма, поэтому в целях уменьшения временных затрат стоит распараллелить эту часть метода.

Во второй главе будут описаны два параллельных алгоритма.

\section*{Вывод}
\addcontentsline{toc}{section}{Вывод}
\qquadБыл рассмотрен общий принцип работы алгоритма Винограда, а также возможные способы его распараллеливания.