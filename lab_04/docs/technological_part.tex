В данном разделе будут приведены листинги функций разрабатываемых алгоритмов и проведено их тестирование.

\section{Выбранный язык программирования}
\qquadДля выполнения этой лабораторной работы был выбран язык программирования C++, так как есть большой навык работы с ним и с подключаемыми библиотеками, которые также использовались для проведения тестирования и замеров. Для реализации параллельных алгоритмов использовались библиотеки $thread$ \cite{thread}, $mutex$ \cite{mutex}. 

Использованная среда разработки - Visual Studio \cite{Visual}.

\section{Листинг кода}
\qquadНиже представлены Листиги \ref{code1}, \ref{code2}, \ref{code3} функций, реализующих последовательный алгоритм Винограда и его модификации в виде распараллеленных алгоритмов.\\

\begin{lstlisting}[label=code1, caption = Последовательный алгоритм Винограда, captionpos=b]
#include "winograd_mult.h"

matrix_t winograd_mult(matrix_t a, matrix_t b, int m, int n, int q)
{
	arr_t mulV = create_array(q);
	int temp = n - 1;
	double mulH, buf;
	
	matrix_t c = create_matrix(m, q);
	
	for (int i = 0; i < q; i++)
	{
		buf = 0;
		for (int k = 1; k < n; k += 2)
		buf += b[k][i] * b[k - 1][i];
		mulV[i] = buf;
	}
	
	for (int i = 0; i < m; i++)
	{
		mulH = 0;
		for (int k = 1; k < n; k += 2)
			mulH += a[i][k] * a[i][k - 1];
		
		for (int j = 0; j < q; j++)
		{
			buf = -(mulH + mulV[j]);
			for (int k = 1, t = 0; k < n; k += 2, t += 2)
				buf += (a[i][k] + b[t][j]) * (a[i][t] + b[k][j]);
			c[i][j] = buf;
			
			if (n % 2)
				c[i][j] += a[i][temp] * b[temp][j];
		}
	}
	
	free_array(&mulV);
	
	return c;
}
\end{lstlisting}

\begin{lstlisting}[label=code2, caption = Распараллеленный алгоритм Винограда (1 вариант), captionpos=b]
#include "parall_winograd.h"

mutex set_mutex;


void process_threads_1(matrix_t& c, int start, matrix_t& a, matrix_t& b, int m, int n, int q, arr_t mulV, int step)
{
	double mulH, buf;
	int temp = n - 1;
	
	for (int i = start; i < m; i += step)
	{
		mulH = 0;
		for (int k = 1; k < n; k += 2)
			mulH += a[i][k] * a[i][k - 1];
		
		for (int j = 0; j < q; j++)
		{
			buf = -(mulH + mulV[j]);
			for (int k = 1, t = 0; k < n; k += 2, t += 2)
				buf += (a[i][k] + b[t][j]) * (a[i][t] + b[k][j]);
			
			set_mutex.lock();
			c[i][j] = buf;
			set_mutex.unlock();
			
			if (n % 2)
			{
				set_mutex.lock();
				c[i][j] += a[i][temp] * b[temp][j];
				set_mutex.unlock();
			}
		}
	}
}

matrix_t parall_winograd_mult_1(matrix_t a, matrix_t b, int m, int n, int q, int num_thread)
{
	arr_t mulV = create_array(q);
	double buf;
	
	matrix_t c = create_matrix(m, q);
	vector<thread> thread_arr;
	
	for (int i = 0; i < q; i++)
	{
		buf = 0;
		for (int k = 1; k < n; k += 2)
			buf += b[k][i] * b[k - 1][i];
		mulV[i] = buf;
	}
	
	for (int i = 0; i < num_thread; i++)
		thread_arr.push_back(thread(process_threads_1, ref(c), i, ref(a), ref(b), m, n, q, mulV, num_thread));
	
	for (int i = 0; i < num_thread; i++)
		thread_arr[i].join();
	
	free_array(&mulV);
	
	return c;
}

\end{lstlisting}

\begin{lstlisting}[label=code3, caption = Распараллеленный алгоритм Винограда (2 вариант), captionpos=b]
#include "parall_winograd.h"

mutex set_mutex;


void process_threads_2(matrix_t& c, int start, matrix_t& a, matrix_t& b, int m, int n, int q, arr_t mulV, int step)
{
	double mulH, buf;
	int temp = n - 1;
	
	for (int i = 0; i < m; i++)
	{
		mulH = 0;
		for (int k = 1; k < n; k += 2)
			mulH += a[i][k] * a[i][k - 1];
		
		for (int j = start; j < q; j += step)
		{
			buf = -(mulH + mulV[j]);
			for (int k = 1, t = 0; k < n; k += 2, t += 2)
				buf += (a[i][k] + b[t][j]) * (a[i][t] + b[k][j]);
			
			set_mutex.lock();
			c[i][j] = buf;
			set_mutex.unlock();
			
			if (n % 2)
			{
				set_mutex.lock();
				c[i][j] += a[i][temp] * b[temp][j];
				set_mutex.unlock();
			}
		}
	}
}

matrix_t parall_winograd_mult_2(matrix_t a, matrix_t b, int m, int n, int q, int num_thread)
{
	arr_t mulV = create_array(q);
	double buf;
	
	matrix_t c = create_matrix(m, q);
	vector<thread> thread_arr;
	
	for (int i = 0; i < q; i++)
	{
		buf = 0;
		for (int k = 1; k < n; k += 2)
			buf += b[k][i] * b[k - 1][i];
		mulV[i] = buf;
	}
	
	for (int i = 0; i < num_thread; i++)
		thread_arr.push_back(thread(process_threads_2, ref(c), i, ref(a), ref(b), m, n, q, mulV, num_thread));
	
	for (int i = 0; i < num_thread; i++)
		thread_arr[i].join();
	
	free_array(&mulV);
	
	return c;
}
\end{lstlisting}


\section{Результаты тестов}
\qquadДля тестирования были написаны функции, проверяющие, согласно заготовкам выше, случаи. Выводы о корректности работы делаются на основе сравнения результатов.\\

\textbf{Все тесты пройдены успешно.} На Листинге \ref{code_test} представлены сами тесты.\\

\begin{lstlisting}[label=code_test, caption = Тесты, captionpos=b]
#include "tests.h"

double PCFreq = 0.0;
__int64 CounterStart = 0;

bool mult_cmp(matrix_t a, matrix_t b, int m, int n, int q, int num_thread = 5)
{
	matrix_t c1 = winograd_mult(a, b, m, n, q);
	matrix_t c2 = parall_winograd_mult_1(a, b, m, n, q, num_thread);
	
	bool res = cmp_matrix(c1, c2, m, q);
	
	free_matrix(&c1, m, q);
	free_matrix(&c2, m, q);
	
	return res;
}

// Матрицы размером 1 х 1
void test_size_1_1()
{
	int n = 1;
	matrix_t a = create_matrix(n, n);
	matrix_t b = create_matrix(n, n);
	
	a[0][0] = 15;
	b[0][0] = -7;
	
	if (!mult_cmp(a, b, n, n, n))
	{
		cout << endl << __FUNCTION__ << " FAILED" << endl;
		free_matrix(&a, n, n);
		free_matrix(&b, n, n);
		return;
	}
	
	free_matrix(&a, n, n);
	free_matrix(&b, n, n);
	
	cout << endl << __FUNCTION__ << " OK" << endl;
}

// Один поток
void test_one_thread()
{
	int m[] = { 2, 6, 10 };
	int n[] = { 1, 4, 7 };
	int q[] = { 3, 4, 8 };
	int num_thread = 1;
	
	for (int i = 0; i < sizeof(n) / sizeof(n[0]); i++)
	{
		matrix_t a = random_fill_matrix(m[i], n[i]);
		matrix_t b = random_fill_matrix(n[i], q[i]);
		
		if (!mult_cmp(a, b, m[i], n[i], q[i], num_thread))
		{
			cout << endl << __FUNCTION__ << " FAILED" << endl;
			free_matrix(&a, m[i], n[i]);
			free_matrix(&b, n[i], q[i]);
			return;
		}
		free_matrix(&a, m[i], n[i]);
		free_matrix(&b, n[i], q[i]);
		
		cout << endl << __FUNCTION__ << " OK" << endl;
	}
}

// Потоков меньше, чем значения M, N, Q
void test_less_mnq()
{
	int m[] = { 4, 8, 10 };
	int n[] = { 3, 5, 12 };
	int q[] = { 3, 6, 8 };
	int num_thread[] = { 2, 4, 2 };
	
	for (int i = 0; i < sizeof(n) / sizeof(n[0]); i++)
	{
		matrix_t a = random_fill_matrix(m[i], n[i]);
		matrix_t b = random_fill_matrix(n[i], q[i]);
		
		if (!mult_cmp(a, b, m[i], n[i], q[i], num_thread[i]))
		{
			cout << endl << __FUNCTION__ << " FAILED" << endl;
			free_matrix(&a, m[i], n[i]);
			free_matrix(&b, n[i], q[i]);
			return;
		}
		free_matrix(&a, m[i], n[i]);
		free_matrix(&b, n[i], q[i]);
		
		cout << endl << __FUNCTION__ << " OK" << endl;
	}
}

// Потоков больше, чем значения M, N, Q
void test_more_mnq()
{
	int m[] = { 4, 8, 10 };
	int n[] = { 3, 5, 12 };
	int q[] = { 3, 6, 8 };
	int num_thread[] = { 7, 10, 15 };
	
	for (int i = 0; i < sizeof(n) / sizeof(n[0]); i++)
	{
		matrix_t a = random_fill_matrix(m[i], n[i]);
		matrix_t b = random_fill_matrix(n[i], q[i]);
		
		if (!mult_cmp(a, b, m[i], n[i], q[i], num_thread[i]))
		{
			cout << endl << __FUNCTION__ << " FAILED" << endl;
			free_matrix(&a, m[i], n[i]);
			free_matrix(&b, n[i], q[i]);
			return;
		}
		free_matrix(&a, m[i], n[i]);
		free_matrix(&b, n[i], q[i]);
		
		cout << endl << __FUNCTION__ << " OK" << endl;
	}
}

// Умножение одних и тех же матриц на разном числе потоков
void test_dif_num_threads()
{
	int m = 20, n = 17, q = 23;
	int num_thread[] = { 1, 2, 5, 8, 10, 14, 16, 19, 20, 24, 26, 30 };
	
	for (int i = 0; i < sizeof(num_thread) / sizeof(num_thread[0]); i++)
	{
		matrix_t a = random_fill_matrix(m, n);
		matrix_t b = random_fill_matrix(n, q);
		
		if (!mult_cmp(a, b, m, n, q, num_thread[i]))
		{
			cout << endl << __FUNCTION__ << " FAILED" << endl;
			free_matrix(&a, m, n);
			free_matrix(&b, n, q);
			return;
		}
		free_matrix(&a, m, n);
		free_matrix(&b, n, q);
		
		cout << endl << __FUNCTION__ << " OK" << endl;
	}
}

void run_tests()
{
	test_size_1_1();
	test_one_thread();
	test_less_mnq();
	test_more_mnq();
	test_dif_num_threads();
}
\end{lstlisting}

\section{Оценка времени}
\qquadПроцессорное время измеряется с помощью функции QueryPerformanceCounter библиотеки windows.h \cite{Query}. Осуществление замеров показано ниже (Листинг \ref{code_time}).\\

\begin{lstlisting}[label=code_time, caption = Замеры процессорного времени, captionpos=b]
void test_range(int n, vector<int>& num_threads)
{
	for (int key : num_threads)
	{
		cout << endl << endl << "Размер тестируемых матриц: " << n << "x" << n << endl;
		cout << "Количество потоков: " << key;
		
		cout << endl << "-----Winograd(improved)-----" << endl;
		test_time_cons(winograd_mult, n);
		cout << endl << "-----Parallel Winograd(1)-----" << endl;
		test_time_parall(parall_winograd_mult_1, n, key);
		cout << endl << "-----Parallel Winograd(2)-----" << endl;
		test_time_parall(parall_winograd_mult_2, n, key);
	}
}

void start_measuring()
{
	LARGE_INTEGER li;
	QueryPerformanceFrequency(&li);
	
	PCFreq = double(li.QuadPart) / 1000;
	
	QueryPerformanceCounter(&li);
	CounterStart = li.QuadPart;
}

double get_measured()
{
	LARGE_INTEGER li;
	QueryPerformanceCounter(&li);
	
	return double(li.QuadPart - CounterStart) / PCFreq;
}

// Замеры процессорного времени
void test_time_parall(matrix_t(*f)(matrix_t, matrix_t, int, int, int, int), int n, int num_threads)
{
	matrix_t a = random_fill_matrix(n, n);
	matrix_t b = random_fill_matrix(n, n);
	matrix_t c;
	
	int num = 0;
	start_measuring();
	
	while (get_measured() < 3 * 1000)
	{
		c = f(a, b, n, n, n, num_threads);
		free_matrix(&c, n, n);
		num++;
	}
	
	double t = get_measured() / 1000;
	cout << "Выполнено " << num << " операций за " << t << " секунд" << endl;
	cout << "Время: " << t / num << endl;
	
	free_matrix(&a, n, n);
	free_matrix(&b, n, n);
}

void test_time_cons(matrix_t(*f)(matrix_t, matrix_t, int, int, int), int n)
{
	matrix_t a = random_fill_matrix(n, n);
	matrix_t b = random_fill_matrix(n, n);
	matrix_t c;
	
	int num = 0;
	start_measuring();
	
	while (get_measured() < 3 * 1000)
	{
		c = f(a, b, n, n, n);
		free_matrix(&c, n, n);
		num++;
	}
	
	double t = get_measured() / 1000;
	cout << "Выполнено " << num << " операций за " << t << " секунд" << endl;
	cout << "Время: " << t / num << endl;
	
	free_matrix(&a, n, n);
	free_matrix(&b, n, n);
}
\end{lstlisting}

\section*{Вывод}
\addcontentsline{toc}{section}{Вывод}
\qquadБыли разработаны функции, реализующие текущие алгоритмы, приведены листинги кода каждой из них, а также листинги тестовых функций, направленных на проверку корректности их работы, и замеров процессорного времени.
