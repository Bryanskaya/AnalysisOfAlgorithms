Проведём замеры процессорного времени, которое затрачивается каждым алгоритмом на поиск ключа, найдём масимальное, минимальное и среднее и сравним полученные результаты.

\section{Характеристики ПК}
\qquadПри проведении замеров времени использовался компьютер, имеющий следующие характеристики:
\begin{itemize}
	\item OC - Windows 10 Pro
	\item Процессор - Inter Core i7 10510U (1800 МГц)
	\item Объём ОЗУ - 16 Гб
\end{itemize}

\section{Измерения}
\qquadДля проведения замеров процессорного времени использовался словарь в 1200 элементов. Его содержимое генерируется случайным образом, и программно обеспечивается уникальность всех ключей. Находится время, затрачиваемое на поиск каждого ключа из словаря, а также несуществующего ключа. \\

Каждый замер проводится 5 раз для получения более точного среднего результата. Выделяется максимальное, минимальное и среднее время.\\

В \hyperref[table_4_1]{таблице 4.1} представлены результаты замеров процессорного времени работы реализаций алгоритмов (в сек).

\begin{table}[ph] \label{table_4_1}
	\caption{Результаты измерений на словаре в 1200 элементов}
	\centering
	\begin{tabular}{|c|c|c|c|}
		\hline
		Алгоритм &Поиск & В упорядоченном & Поиск полным\\
		/    &полным &словаре &перебором с\\
		Время    &перебором &двоичным поиском &использованием сегментов \\
		\hline
		Максимальное & $1.68\cdot10^{-4}$ & $1.115\cdot10^{-5}$ & $8.763\cdot10^{-5}$ \\
		\hline
		Минимальное & $3.689\cdot10^{-6}$ & $4.121\cdot10^{-6}$ & $6.918\cdot10^{-6}$ \\
		\hline
		Среднее & $8.068\cdot10^{-5}$ & $8.071\cdot10^{-6}$ & $2.690\cdot10^{-5}$ \\
		\hline
		Несуществующий ключ & $1.527\cdot10^{-4}$ & $9.642\cdot10^{-6}$ & $5.285\cdot10^{-5}$ \\
		\hline
	\end{tabular}
\end{table}

Согласно полученным данным можно сделать следующие \textbf{выводы}:
\begin{itemize}
	\item алгоритм поиска полным перебором показывает наибольшее максимальное время по сравнению с другими рассматриваемыми алгоритмами (на один порядок больше, чем два других алгоритма);
	\item с другой стороны, алгоритм полного перебора демонстрирует наименьшее минимальное время среди остальных (алгоритм с двоичным поиском больше в 1.1 раза, а алгоритм с сегментами в 1.88);
	\item минимальное время у всех трёх алгоритмов имеет одинаковый порядок;
	\item наименьшее среднее время показывает алгоритм с использованием двоичного поиска;
	\item время, затрачиваемое на поиск несуществующего ключа, примерно равно максимальному значению (наибольшая разница наблюдается у алгоритма с сегментами, объясняется тем, что сегменты значительно сужают область поиска, тем самым, не нужно рассматривать все элементы словаря).
\end{itemize}

\section*{Вывод}
\addcontentsline{toc}{section}{Вывод}
\qquadПроведены замеры процессорного времени, и на основе полученных данных были составлены сравнительные таблицы, описывающие время, которое каждый из алгоритмов затрачивает на поиск. В результате анализа получившихся таблиц были сделаны выводы, приведённые выше.
