В данном разделе будут приведены листинги функций разрабатываемых алгоритмов поиска ключа.

\section{Выбранный язык программирования}
\qquadДля выполнения этой лабораторной работы был выбран язык программирования C++, так как есть большой навык работы с ним и с подключаемыми библиотеками, которые также использовались для проведения тестирования и замеров. \cite{c_plus_plus}

Использованная среда разработки - Visual Studio. \cite{Visual}

\section{Листинг кода}
\qquadНиже представлены Листиги \ref{code1}, \ref{code2}, \ref{code3} функций, реализующих алгоритмы поиска ключей в словаре. В Листинге \ref{code4} показаны собственные типы, используемые в написании функций. А в Листингах \ref{code5}, \ref{code6} приведены вспомогательные функции сортировки и разделения словаря по сегментам.

\begin{lstlisting}[label=code1, caption = Поиск полным перебором]
s_p full_search(string key, sp_arr& data)
{
	int len = data.size();
	
	for (int i = 0; i < len; i++)
		if (data[i].key == key)
			return data[i];
	return not_found();
}
\end{lstlisting}

\begin{lstlisting}[label=code2, caption = Поиск в упорядоченном словаре двоичным поиском]
s_p ordered_search(string key, sp_arr& data)
{
	int left = 0, right = data.size() - 1;
	
	while (right >= left) {
		int middle = (left + right) / 2;   
		
		if (data[middle].key < key)
			left = middle + 1;
		else if (data[middle].key > key)
			right = middle - 1;
		else
			return data[middle];
	}
	return not_found();
}
\end{lstlisting}

\begin{lstlisting}[label=code3, caption = Поиск полным перебором с использованием сегментов]
s_p segment_search(string key, sgm_arr& data)
{
	string part_key = key.substr(key.rfind('.') + 1);;
	
	for (size_t i = 0; i < data.size(); i++)
	{
		for (size_t j = 0; j < data[i].key.size(); j++)
			if (data[i].key[j] == part_key)
			{
				sp_arr& arr = data[i].value;
				return full_search(key, arr);
			}
	}
	
	return not_found();
}
\end{lstlisting}

\begin{lstlisting}[label=code4, caption = Используемые типы]
using s_p = struct
{
	string key;
	string value;
};
using sp_arr = vector<s_p>;

using sgm = struct
{
	vector<string> key;
	sp_arr value;
};
using sgm_arr = vector<sgm>;
\end{lstlisting}

\begin{lstlisting}[label=code5, caption = Функция сортировки словаря]
void sort_arr(sp_arr& data)
{
	s_p temp;
	
	for (size_t i = 0; i < data.size() - 1; i++)
		for (size_t j = 0; j < data.size() - i - 1; j++)
			if (data[j].key > data[j + 1].key) {
				temp = data[j];
				data[j] = data[j + 1];
				data[j + 1] = temp;
			}
}
\end{lstlisting}

\begin{lstlisting}[label=code6, caption = Функция разделения словаря по сегментам]
sgm_arr split_segments(sp_arr& data)
{
	sgm_arr arr;
	vector<string> keys_1 = { "ru", "com", "io" }, keys_2 = { "net", "biz", "org", "info" };
	string part_key;
	int flag = 0;
	
	for (size_t i = 0; i < keys_1.size(); i++)
	{
		sgm temp_1;
		
		temp_1.key.push_back(keys_1[i]);
		arr.push_back(temp_1);
	}
	
	for (size_t i = 0; i < keys_2.size() - 1; i += 2)
	{
		sgm temp_2;
		
		temp_2.key.push_back(keys_2[i]);
		temp_2.key.push_back(keys_2[i + 1]);
		
		arr.push_back(temp_2);
	}
	
	for (size_t i = 0; i < data.size(); i++)
	{
		flag = 0;
		part_key = data[i].key.substr(data[i].key.rfind('.') + 1);
		
		for (size_t j = 0; j < arr.size(); j++)
		{
			for (size_t k = 0; k < arr[j].key.size(); k++)
			{
				if (part_key == arr[j].key[k])
				{
					arr[j].value.push_back(data[i]);
					flag = 1;
					break;
				}
			}
			if (flag)
			break;
		}
	}
	
	return arr;
}
\end{lstlisting}

\section{Результаты тестов}
\qquadДля тестирования были написаны функции, проверяющие, согласно заготовкам выше, случаи. Выводы о корректности работы делаются на основе сравнения результатов.\\

\textbf{Все тесты пройдены успешно.} Сами тесты представлены ниже (Листинг \ref{code_test}).\\

\begin{lstlisting}[label=code_test, caption = Тесты]
bool is_equal(s_p res, s_p ans)
{
	if (res.key == ans.key && res.value == ans.value)
		return true;
	return false;
}

// Нахождение первого элемента словаря
void test_first_key(sp_arr& data)
{
	sgm_arr sgm_data;
	
	if (!is_equal(full_search(data[0].key, data), data[0]))
		cout << endl << __FUNCTION__ << " full_search " << ":\tFAILED\n";
	else
		cout << endl << __FUNCTION__ << "\tfull_search " << ":\tSUCCESS\n";
	
	sort_arr(data);
	if (!is_equal(ordered_search(data[0].key, data), data[0]))
		cout << __FUNCTION__ << " ordered_search " << ":\tFAILED\n";
	else
		cout << __FUNCTION__ << "\tordered_search " << ":\tSUCCESS\n";
	
	sgm_data = split_segments(data);
	if (!is_equal(segment_search(data[0].key, sgm_data), data[0]))
		cout << __FUNCTION__ << " segment_search " << ":\tFAILED\n";
	else
		cout << __FUNCTION__ << "\tsegment_search " << ":\tSUCCESS\n";
}

// Нахождение каждого 100 элемента словаря
void test_each_100_key(sp_arr& data)
{
	sgm_arr sgm_data;
	int flag = 1;
	
	for (size_t i = 0; i < data.size(); i += 100)
		if (!is_equal(full_search(data[i].key, data), data[i]))
		{
			cout << endl << __FUNCTION__ << " full_search " << ":\tFAILED\n";
			flag = 0;
			break;
		}
	if (flag)
		cout << endl << __FUNCTION__ << "\tfull_search " << ":\tSUCCESS\n";
	
	flag = 1;
	sort_arr(data);
	for (size_t i = 0; i < data.size(); i += 100)
		if (!is_equal(ordered_search(data[i].key, data), data[i]))
		{
			cout << __FUNCTION__ << " ordered_search " << ":\tFAILED\n";
			flag = 0;
			break;
		}
	if (flag)
		cout << __FUNCTION__ << "\tordered_search " << ":\tSUCCESS\n";
	
	flag = 1;
	sgm_data = split_segments(data);
	for (size_t i = 0; i < data.size(); i += 100)
		if (!is_equal(segment_search(data[i].key, sgm_data), data[i]))
		{
			cout << __FUNCTION__ << " segment_search " << ":\tFAILED\n";
			flag = 0;
			break;
		}
	if (flag)
	cout << __FUNCTION__ << "\tsegment_search " << ":\tSUCCESS\n";
}

// Нахождение последнего элемента словаря
void test_last_key(sp_arr& data)
{
	sgm_arr sgm_data;
	
	if (!is_equal(full_search(data[data.size() - 1].key, data), data[data.size() - 1]))
		cout << endl << __FUNCTION__ << " full_search " << ":\tFAILED\n";
	else
		cout << endl << __FUNCTION__ << "\tfull_search " << ":\tSUCCESS\n";
	
	sort_arr(data);
	if (!is_equal(ordered_search(data[data.size() - 1].key, data), data[data.size() - 1]))
		cout << __FUNCTION__ << " ordered_search " << ":\tFAILED\n";
	else
		cout << __FUNCTION__ << "\tordered_search " << ":\tSUCCESS\n";
	
	sgm_data = split_segments(data);
	if (!is_equal(segment_search(data[data.size() - 1].key, sgm_data), data[data.size() - 1]))
		cout << __FUNCTION__ << " segment_search " << ":\tFAILED\n";
	else
		cout << __FUNCTION__ << "\tsegment_search " << ":\tSUCCESS\n";
}

// Нахождение несуществующего элемента словаря
void test_not_exist_key(sp_arr& data)
{
	sgm_arr sgm_data;
	s_p key_not_exst;
	
	key_not_exst.key = "123345";
	key_not_exst.value = "000";
	
	if (!is_equal(full_search(key_not_exst.key, data), not_found()))
		cout << endl << __FUNCTION__ << " full_search " << ":\tFAILED\n";
	else
		cout << endl << __FUNCTION__ << "\tfull_search " << ":\tSUCCESS\n";
	
	sort_arr(data);
	if (!is_equal(ordered_search(key_not_exst.key, data), not_found()))
		cout << __FUNCTION__ << " ordered_search " << ":\tFAILED\n";
	else
		cout << __FUNCTION__ << "\tordered_search " << ":\tSUCCESS\n";
	
	sgm_data = split_segments(data);
	if (!is_equal(segment_search(key_not_exst.key, sgm_data), not_found()))
		cout << __FUNCTION__ << " segment_search " << ":\tFAILED\n";
	else
		cout << __FUNCTION__ << "\tsegment_search " << ":\tSUCCESS\n";
}

void run_tests(sp_arr& data)
{
	cout << "----- START TESTING -----" << endl;
	
	test_first_key(data);
	test_each_100_key(data);
	test_last_key(data);
	test_not_exist_key(data);
	
	cout << endl << "----- FINISHED -----" << endl;
}
\end{lstlisting}	

\section{Оценка времени}
\qquadПроцессорное время измеряется с помощью функции QueryPerformanceCounter библиотеки windows.h. \cite{Query} Осуществление замеров показано ниже (Листинг \ref{code_time}).

\begin{lstlisting}[label=code_time, caption = Замеры процессорного времени]
double PCFreq = 0.0;
__int64 CounterStart = 0;

void start_measuring()
{
	LARGE_INTEGER li;
	QueryPerformanceFrequency(&li);
	
	PCFreq = double(li.QuadPart) / 1000;
	
	QueryPerformanceCounter(&li);
	CounterStart = li.QuadPart;
}

double get_measured()
{
	LARGE_INTEGER li;
	QueryPerformanceCounter(&li);
	
	return double(li.QuadPart - CounterStart) / PCFreq;
}

void measure_time(func_t f, sp_arr& data)
{
	double min_t = 1e5, max_t = -1, avg_t = 0, t;
	string key, not_exist_key = "123456";
	int count;
	
	for (size_t i = 0; i < data.size(); i++)
	{
		key = data[i].key;
		count = 0;
		
		start_measuring();
		while (get_measured() < 0.07 * 1000)
		{
			f(key, data);
			count++;
		}
		
		t = get_measured() / 1000 / count;
		
		cout << i + 1 << " " << t << endl;
		
		if (min_t > t)
			min_t = t;
		if (max_t < t)
			max_t = t;
		avg_t += t;
	}
	
	avg_t /= data.size();
	
	count = 0;
	start_measuring();
	while (get_measured() < 0.07 * 1000)
	{
		f(not_exist_key, data);
		count++;
	}
	
	t = get_measured() / 1000 / count;
	cout << "NOT EXISTS" << " " << t << endl;
	
	cout << "\nМаксимальное время:\t" << max_t << endl;
	cout << "Минимальное время:\t" << min_t << endl;
	cout << "Среднее время:\t\t" << avg_t << endl;
}

void measure_time_sgm(func_sgm_t f, sgm_arr& data)
{
	double min_t = 1e5, max_t = -1, avg_t = 0, t;
	string key, not_exist_key = "123456";
	int count, ind = 0;
	
	for (size_t i = 0; i < data.size(); i++)
	{
		for (size_t j = 0; j < data[i].value.size(); j++)
		{
			ind++;
			key = data[i].value[j].key;
			count = 0;
			
			start_measuring();
			while (get_measured() < 0.03 * 1000)
			{
				f(key, data);
				count++;
			}
			
			t = get_measured() / 1000 / count;
			
			cout << ind << " " << t << endl;
			
			if (min_t > t)
				min_t = t;
			if (max_t < t)
				max_t = t;
			avg_t += t;
		}
	}
	
	avg_t /= data.size();
	
	count = 0;
	start_measuring();
	while (get_measured() < 0.07 * 1000)
	{
		f(not_exist_key, data);
		count++;
	}
	
	t = get_measured() / 1000;
	cout << "NOT EXISTS" << " " << t << endl;
	
	cout << "\nМаксимальное время:\t" << max_t << endl;
	cout << "Минимальное время:\t" << min_t << endl;
	cout << "Среднее время:\t\t" << avg_t << endl;
}
\end{lstlisting}

\section*{Вывод}
\addcontentsline{toc}{section}{Вывод}
\qquadТаким образом, приведены листинги кода каждой из функций, реализующих алгоритмы поиска ключа в словаре, а также листинг тестовых функций, направленных на проверку корректности их работы.