В этой лабораторной работе будут рассматриваться различные алгоритмы поиска ключа в словаре и проводиться их сравнение по затрачиваемому времени. \\

\textbf{Словарь} -- структура данных, представляющая собой специальным образом организованный набор элементов хранимых данных. \\

Каждый элемент словаря состоит из двух объектов: ключа и значения (также называется сопутствующим данным). Значения ключей -- уникальны, двух одинаковых в словаре не может быть. Для того, чтобы получить значение, нужно сначала дойти до соответствующего ключа.\\

Словари могут быть очень большими по объёму, и поиск нужного слова порой требует значительных временных затрат, поэтому необходимо найти такой алгоритм, который бы решал поставленную задачу за меньший промежуток времени для всех возможных случаев расположения ключа.\\

В данной лабораторной работе будут использоваться такие алгоритмы, как полный перебор, поиск в упорядоченном словаре двоичным поиском и поиск полным перебором с сегментацией.

