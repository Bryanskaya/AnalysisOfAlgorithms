В этом разделе будут поставлены цель и основные задачи лабораторной работы, которые будут решаться по мере её выполнения.

\section{Цель и задачи}
\qquad\textbf{Цель} данной работы: реализовать и сравнить алгоритмы поиска в словаре.\\

Для достижения поставленной цели необходимо решить следующий ряд \textbf{задач}:
\begin{enumerate}
	\item[1)] дать описание используемого словаря;
	\item[2)] описать алгоритмы поиска;
	\item[3)] реализовать все рассмотренные алгоритмы;
	\item[4)] провести замеры процессорного времени работы алгоритмов;
	\item[5)] найти минимальное/максимальное/среднее время и время работы при несуществующем ключе.
\end{enumerate}

\section{Описание словаря}
\qquadДля этой лабораторной работы был составлен словарь сайтов и паролей к ним, где ключ -- это url сайта, а значение -- пароль.

\section{Используемые алгоритмы поиска}
\qquadБудут использованы следующие алгоритмы поиска ключа.

\subsection{Поиск полным перебором}
\qquadЭтот алгоритм один из самых простых в реализации. В нём используется метод полного перебора, последовательно просматриваются все элементы словаря до тех пор, пока не найдётся соответсвие или пока не проанализуются все возможные варианты. \cite{Kormen}

\subsection{Поиск в упорядоченном словаре двоичным поиском}
\qquadДля этого алгоритма необходимо в качестве подготовительного этапа сначала упорядочить данные. Затем на получившемся наборе производится двоичный поиск ключа, использующий дробление массива на половины. \\

На каждом шаге алгоритма массив данных делится пополам, и дальнейшая работа производится с той частью, где потенциально должно находиться искомое значение. \cite{Tardos}

\subsection{Поиск полным перебором с использованием сегментов}
\qquadВ этом алгоритме производится разбиение исходных данных на сегменты, которые объединены общим признаком. Помимо общих черт, в качестве принципа разделения можно взять частоту обращения к каждому элементу, и на основе анализа частот выделить группы.\\

Для того, чтобы осуществить поиск заданного ключа, необходимо сначала найти сегмент, в котором он может потенциально находится, а затем в этом сегменте попытаться его найти. 

\section*{Вывод}
\addcontentsline{toc}{section}{Вывод}
\qquadБыли поставлены цель и задачи текущей лабораторной работы, также дано краткое описание рассматриваемых алгоритмов.