В ходе лабораторной работы была достигнута поставленная цель, а именно, исследованы, реализованы и сопоставлены алгоритмы поиска ключа в словаре.\\

В процессе выполнения были решены все задачи. Описаны словарь, все рассматриваемые алгоритмы. Все проработанные алгоритмы реализованы, кроме того, были проведены замеры процессорного времени работы поиска и проведён сравнительный анализ, сделаны выводы.\\

По результатам замеров процессорного времени сделаны следующие заключения.
\begin{itemize}
	\item алгоритм поиска полным перебором показывает как худшее максимальное время, так и лучшее минимальное по сравнению с другими рассматриваемыми алгоритмами;
	\item минимальное время у всех трёх алгоритмов одинакового порядка;
	\item алгоритм с использованием двоичного поиска имеет наименьшее среднее время;
	\item время, затрачиваемое на поиск несуществующего ключа, примерно равно максимальному значению (наибольшая разница наблюдается у алгоритма с сегментами, объясняется тем, что сегменты значительно сужают область поиска, тем самым, не нужно рассматривать все элементы словаря).
\end{itemize}