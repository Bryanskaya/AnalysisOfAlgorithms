\textbf{Цель} данной работы – оценить трудоёмкость алгоритмов умножения матриц и получить практический навык оптимизации алгоритмов.\\

Для достижения поставленной цели необходимо решить ряд следующих \textbf{задач}:
\begin{enumerate}
\item[1)] дать математическое описание;
\item[2)] описать алгоритмы умножения матриц;
\item[3)] дать теоретическую оценку трудоёмкости алгоритмов;
\item[4)] реализовать эти алгоритмы ;
\item[5)] провести замеры процессорного времени работы алгоритмов на материале серии экспериментов;
\item[6)] провести сравнительный анализ алгоритмов.
\end{enumerate}

Умножение осуществляется над матрицами $A[M \times N]$ и $B[N \times Q]$. Число столбцов первой матрицы должно совпадать с числом строк второй, а таком случае можно осуществлять умножение. Результатом является матрица $C[M \times Q]$, в которой число строк столько же, сколько в первой, а столбцов, столько же, сколько во второй.\\

В основе \textbf{стандартного алгоритма} умножения матриц лежит следующая формула:
\begin{equation}\label{formula1}
	c_{i,j} = \sum_{k=1}^{N}(a_{i,k} \times b_{k,j})
\end{equation}

Существует и другой алгоритм умножения - \textbf{алгоритм Винограда}.
Обозначим строку $A_{i,*}$ как $\overrightarrow{u}$, $B_{*,j}$ как $\overrightarrow{v}$.\\

Пусть $u = (u_1, u_2, u_3, u_4)$ и $v = (v_1, v_2, v_3, v_4)$, тогда их произведение равно\\
\begin{equation}\label{formula2}
u \cdot v = u_1 \cdot v_1 + u_2 \cdot v_2 + u_3 \cdot v_3 + u_4 \cdot v_4
\end{equation}

Выражение (\ref{formula2}) можно преобразовать в следующее: 
\begin{equation}\label{formula3}
	u \cdot v = (u_1 + v_2)\cdot(u_2 + v_1) + (u_3 + v_1)\cdot(u_4 + v_3) - u_1\cdot u_2 - u_3\cdot u_4 - v_1\cdot v_2 - v_3\cdot v_4
\end{equation}

Алгоритм Винограда основывается на раздельной работе со слагаемыми из выражения (\ref{formula3}). 