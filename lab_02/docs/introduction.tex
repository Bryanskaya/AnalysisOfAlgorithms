\textbf{Трудоёмкость алгоритма} - это зависимость стоимости операций от линейного(ых) размера(ов) входа(ов).\\

Модель вычислений трудоёмкости должна учитывать следующие оценки.
\begin{itemize}
	\item[1)] Стоимость базовых операций. К ним относятся: =, +, -, *, /, ==, !=, <, <=, >, >=, \%, +=, -=, *=, /=, [ ], < <, > >. Каждая из операций имеет стоимость равную 1.
	\item[2)] Оценка цикла. Она складывается из стоимости тела, инкремента и сравнения. 
	\item[3)] Оценка условного оператора if. Положим, что стоимость перехода к одной из веток равной 0. В таком случае, общая стоимость складывается из подсчета условия и рассмотрения худшего и лучшего случаев.
\end{itemize}

Оценка характера трудоёмкости даётся по наиболее быстрорастущему слагаемому.\\

В этой лабораторной работе будет оцениваться трудоёмкость алгоритмов умножения матриц.