В этой лабораторной работе будет оцениваться \textbf{трудоёмкость алгоритмов умножения матриц}.\\

\textbf{Трудоёмкость алгоритма} - это зависимость стоимости операций от линейного(ых) размера(ов) входа(ов) \cite{labor_int}.\\

Модель вычислений трудоёмкости должна учитывать:
\begin{itemize}
	\item[1)] стоимость базовых операций. К ним относятся: =, +, -, *, /, ==, !=, <, <=, >, >=, \%, +=, -=, *=, /=, [ ], < <, > >. Каждая из операций имеет стоимость равную 1.
	\item[2)] оценку цикла. Она складывается из стоимости тела, инкремента и сравнения. 
	\item[3)] оценку условного оператора if. Положим, что стоимость перехода к одной из веток равной 0. В таком случае, общая стоимость складывается из подсчета условия и рассмотрения худшего и лучшего случаев.
\end{itemize}

Оценка характера трудоёмкости даётся по наиболее быстрорастущему слагаемому.\\

