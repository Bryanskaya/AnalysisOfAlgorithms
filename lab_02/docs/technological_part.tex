\section{Выбранный язык программирования}
\qquadДля выполнения этой лабораторной работы был выбран язык программирования C++, так как есть большой навык работы с ним и с подключаемыми библиотеками, которые также использовались для проведения тестирования и замеров.\\

Использованная среда разработки - Visual Studio \cite{Visual}.

\section{Листинг кода}
\qquadНиже представлены Листиги 3.1 - 3.3 функций, реализующих алгоритмы поиска расстояний.
\begin{lstlisting}[label=code, caption = Стандартный алгоритм умножения матриц]
	matrix_t standart_mult(matrix_t a, matrix_t b, int m, int n, int q)
	{
		matrix_t c = create_matrix(m, q);
		
		for (int i = 0; i < m; i++)
			for (int j = 0; j < q; j++)
			{
				c[i][j] = 0;
				for (int k = 0; k < n; k++)
					c[i][j] += a[i][k] * b[k][j];
			}
		
		return c;
	}
\end{lstlisting}

\begin{lstlisting}[label=code, caption = Алгоритм Винограда]
matrix_t winograd_mult(matrix_t a, matrix_t b, int m, int n, int q)
{
	arr_t mulH = create_array(m);
	arr_t mulV = create_array(q);
	matrix_t c = create_matrix(m, q);
	
	for (int i = 0; i < m; i++)
	{
		mulH[i] = 0;
		for (int k = 0; k < n / 2; k++)
			mulH[i] = mulH[i] + a[i][2 * k] * a[i][2 * k + 1];
	}
	
	for (int i = 0; i < q; i++)
	{
		mulV[i] = 0;
		for (int k = 0; k < n / 2; k++)
			
			mulV[i] = mulV[i] + b[2 * k][i] * b[2 * k + 1][i];
	}
	
	for (int i = 0; i < m; i++)
		for (int j = 0; j < q; j++)
		{
			c[i][j] = -mulH[i] - mulV[j];
			for (int k = 0; k < n / 2; k++)
				c[i][j] = c[i][j] + (a[i][2 * k] + b[2 * k + 1][j]) * 
						  		(a[i][2 * k + 1] + b[2 * k][j]);
		}
	
	if (n % 2)
		for (int i = 0; i < m; i++)
			for (int j = 0; j < q; j++)
				c[i][j] = c[i][j] + a[i][n - 1] * b[n - 1][j];
	
	return c;
}
\end{lstlisting}

\begin{lstlisting}[label=code, caption = Оптимизированный алгоритм Винограда]
matrix_t winograd_mult(matrix_t a, matrix_t b, int m, int n, int q)
{
	arr_t mulH = create_array(m);
	arr_t mulV = create_array(q);
	double buf;
	
	matrix_t c = create_matrix(m, q);
	
	for (int i = 0; i < m; i++)
	{
		buf = 0;
		for (int k = 1; k < n; k += 2)
			buf += a[i][k] * a[i][k - 1];
		mulH[i] = buf;
	}
	
	for (int i = 0; i < q; i++)
	{
		buf = 0;
		for (int k = 1; k < n; k += 2)
			buf += b[k][i] * b[k - 1][i];
		mulV[i] = buf;
	}
	
	int temp = n - 1;
	int is_odd = n % 2;
	
	for (int i = 0; i < m; i++)
		for (int j = 0; j < q; j++)
		{
			buf = -(mulH[i] + mulV[j]);
			for (int k = 1, t = 0; k < n; k += 2, t +=2)
				buf += (a[i][k] + b[t][j]) * (a[i][t] + b[k][j]);
			c[i][j] = buf;
			
			if (is_odd)
				c[i][j] += a[i][temp] * b[temp][j];
		}
	
	return c;
}
\end{lstlisting}

\section{Результаты тестов}
\qquadДля тестирования были написаны функции, проверяющие, согласно заготовкам выше, случаи. Выводы о корректности работы делаются на основе сравнения результатов.

\textbf{Все тесты пройдены успешно.} Сами тесты представлены ниже (Листинг 3.4).\\

\begin{lstlisting}[label=code, caption = Тесты]
bool mult_cmp(matrix_t a, matrix_t b, int m, int n, int q)
{
	matrix_t c1 = standart_mult(a, b, m, n, q);
	matrix_t c2 = winograd_mult(a, b, m, n, q);
	
	bool res = cmp_matrix(c1, c2, m, q);
	
	free_matrix(&c1, m, q);
	free_matrix(&c2, m, q);
	
	return res;
}

// Матрицы размером 1 х 1
void test_size_1_1()
{
	int n = 1;
	
	matrix_t a = create_matrix(n, n);
	matrix_t b = create_matrix(n, n);
	
	a[0][0] = 15;
	b[0][0] = -7;
	
	if (!mult_cmp(a, b, n, n, n))
	{
		cout << endl << __FUNCTION__ << " FAILED" << endl;
		free_matrix(&a, n, n);
		free_matrix(&b, n, n);
		return;
	}
	
	free_matrix(&a, n, n);
	free_matrix(&b, n, n);
	
	cout << endl << __FUNCTION__ << " OK" << endl;
}

// Квадратные матрицы
void test_square_matr()
{
	int n[] = { 2, 6, 10 };
	
	for (int i = 0; i < sizeof(n) / sizeof(n[0]); i++)
	{
		matrix_t a = random_fill_matrix(n[i], n[i]);
		matrix_t b = random_fill_matrix(n[i], n[i]);
		
		if (!mult_cmp(a, b, n[i], n[i], n[i]))
		{
			cout << endl << __FUNCTION__ << " FAILED" << endl;
			free_matrix(&a, n[i], n[i]);
			free_matrix(&b, n[i], n[i]);
			return;
		}
		
		free_matrix(&a, n[i], n[i]);
		free_matrix(&b, n[i], n[i]);
		
		cout << endl << __FUNCTION__ << " OK" << endl;
	}
}

// Прямоугольные матрицы
void test_rectangulat_matr()
{
	int m[] = { 2, 6, 10 };
	int n[] = { 1, 4, 7 };
	int q[] = { 3, 4, 8 };
	
	for (int i = 0; i < sizeof(n) / sizeof(n[0]); i++)
	{
		matrix_t a = random_fill_matrix(m[i], n[i]);
		matrix_t b = random_fill_matrix(n[i], q[i]);
		
		if (!mult_cmp(a, b, m[i], n[i], q[i]))
		{
			cout << endl << __FUNCTION__ << " FAILED" << endl;
			free_matrix(&a, m[i], n[i]);
			free_matrix(&b, n[i], q[i]);
			return;
		}
		free_matrix(&a, m[i], n[i]);
		free_matrix(&b, n[i], q[i]);
		
		cout << endl << __FUNCTION__ << " OK" << endl;
	}
}

// Матрицы с чётным размером
void test_even_size()
{
	int m[] = { 2, 4 };
	int n[] = { 6, 2 };
	int q[] = { 2, 8 };
	
	for (int i = 0; i < sizeof(n) / sizeof(n[0]); i++)
	{
		matrix_t a = random_fill_matrix(m[i], n[i]);
		matrix_t b = random_fill_matrix(n[i], q[i]);
		
		if (!mult_cmp(a, b, m[i], n[i], q[i]))
		{
			cout << endl << __FUNCTION__ << " FAILED" << endl;
			free_matrix(&a, m[i], n[i]);
			free_matrix(&b, n[i], q[i]);
			return;
		}
		
		free_matrix(&a, m[i], n[i]);
		free_matrix(&b, n[i], q[i]);
		
		cout << endl << __FUNCTION__ << " OK" << endl;
	}
}

// Матрицы с нечётным размером
void test_odd_size()
{
	int m[] = { 3, 3 };
	int n[] = { 3, 1 };
	int q[] = { 5, 7 };
	
	for (int i = 0; i < sizeof(n) / sizeof(n[0]); i++)
	{
		matrix_t a = random_fill_matrix(m[i], n[i]);
		matrix_t b = random_fill_matrix(n[i], q[i]);
		
		if (!mult_cmp(a, b, m[i], n[i], q[i]))
		{
			cout << endl << __FUNCTION__ << " FAILED" << endl;
			free_matrix(&a, m[i], n[i]);
			free_matrix(&b, n[i], q[i]);
			return;
		}
		
		free_matrix(&a, m[i], n[i]);
		free_matrix(&b, n[i], q[i]);
		
		cout << endl << __FUNCTION__ << " OK" << endl;
	}
}

void run_tests()
{
	test_size_1_1();
	test_square_matr();
	test_rectangulat_matr();
	test_even_size();
	test_odd_size();
}
\end{lstlisting}

\section{Оценка трудоёмкости}
\qquadПроизведём оценку трудоёмкости приведённых алгоритмов. Рассмотрим умножение матриц $A[M \times N]$ и $B[N \times Q]$.\\

\textbf{Стандартный алгоритм}\\
$f_{st} = 2 + M(2 + 2 + Q(3 + 2 + 2 + N(2 + 1 + 2 + 2 + 1 + 2)))$\\
$f_{st} = 2 + 4M + 7MQ + 10MNQ$\\

\textbf{Алгоритм Винограда (неоптимизированный)}\\
$f_{w} = 2 + M(2 + 3 + 2 + \frac{N}{2} (12 + 3)) + 
2 + Q(2 + 3 + 2 + \frac{N}{2}(12 + 3)) + 
2 + M(2 + 2 + Q(7 + 2 + 3 + \frac{N}{2}(23 + 3))) + 
1 + $$\left[ 
\begin{array}{ll}
	0, & $л.с.$\\
	2 + M(2 + 2 + Q(13 + 2)), & $х.с.$
\end{array} \right.$\\
$f_{w} = 7 + 11M + 7Q + \frac{15}{2}MN + \frac{15}{2}NQ + 12MQ + 13MNQ +
$$\left[\begin{array}{ll}
	0, & $л.с.$\\
	2 + 4M + 15MQ, & $х.с.$
\end{array} \right.$\\

\textbf{Алгоритм Винограда (оптимизированный)}\\
$f_{wop} = 2 + M(2 + 1 + 2 + \frac{N}{2}(7 + 2) + 2) + 2 + Q(2 + 1 + 2 + \frac{N}{2}(7 + 2) + 2) + 2 + 2 + M(2 + 2 + Q(2 + 4 + 3 + \frac{N}{2}(3 + 12) + 3 + 1 + $$\left[\begin{array}{ll}
	0, & $л.с.$\\
	8, & $х.с.$
\end{array} \right.$))\\
$f_{wop} = 8 + 11M + 7Q + 4.5MN + 4.5NQ + 13MQ + 7.5MNQ +$$\left[\begin{array}{ll}
	0, & $л.с.$\\
	8MQ, & $х.с.$
\end{array} \right.$\\

\section{Оценка времени}
\qquadПроцессорное время измеряется с помощью функции QueryPerformanceCounter библиотеки windows.h \cite{Query}. Осуществление замеров показано ниже (Листинг 3.5).
\begin{lstlisting}[label=code, caption = Замеры процессорного времени]
void test_time(matrix_t(*f)(matrix_t, matrix_t, int, int, int), int n)
{
	matrix_t a = random_fill_matrix(n, n);
	matrix_t b = random_fill_matrix(n, n);
	matrix_t c;
	
	int num = 0;
	start_measuring();
	
	while (get_measured() < 3 * 1000)
	{
		c = f(a, b, n, n, n);
		free_matrix(&c, n, n);
		num++;
	}
	
	double t = get_measured() / 1000;
	cout << "Выполнено " << num << " операций за " << t << " секунд" << endl;
	cout << "Время: " << t / num << endl;
	
	free_matrix(&a, n, n);
	free_matrix(&b, n, n);
}

void test_range(vector<int> &n)
{
	for (int key : n)
	{
		cout << endl << endl << "Размер тестируемых матриц: " << key << "x" << key << endl;
		
		cout << endl << "-----Standart-----" << endl;
		test_time(standart_mult, key);
		cout << endl << "-----Winograd-----" << endl;
		test_time(winograd_mult, key);
		cout << endl << "-----Winograd(improved)-----" << endl;
		test_time(winograd_opt_mult, key);
	}
}
\end{lstlisting}