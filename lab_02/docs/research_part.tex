Для проведения замеров процессорного времени использовались квадратные матрицы. Их содержимое генерируется случайным образом. Было проведено две серии экспериментов, ориентированных на выявление чувствительности алгоритмов к чётным и нечётным значениям N.
\begin{itemize}
	\item ${100, 200, 300, 400, 500}$
	\item ${101, 201, 301, 401, 501}$
\end{itemize} 

Каждый замер проводится 5 раз для получения более точного среднего результата.

В \hyperref[table_4_1]{таблице 4.1} и \hyperref[table_4_2]{таблице 4.2} представлены результаты замеров процессорного времени работы реализаций алгоритмов (в сек).

\begin{table}[ph] \label{table_4_1}
	\caption{Результаты измерений (чётная размерность)}
	\centering
	\begin{tabular}{|c|c|c|c|c|c|c|}
		\hline
		Размер n&&&&&&\\
		/    &50 &100 & 200 & 300 & 400 & 500\\
		Алгоритм    &&&&&&\\
		\hline
		Стандартный & $5.1*10^{-4}$ & $4.1*10^{-3}$ & 0.037 & 0.133 & 0.322 & 0.777\\
		\hline
		Виноград & $3.5*10^{-4}$ & $2.9*10^{-3}$ & 0.027 & 0.096 & 0.237 & 0.559\\
		\hline
		Виноград & $3.4*10^{-4}$ & $2.5*10^{-3}$ & 0.024 & 0.084 & 0.207 & 0.474\\
		(оптимизированный) &&&&&&\\
		\hline
	\end{tabular}
\end{table}

\begin{table}[ph] \label{table_4_2}
	\caption{Результаты измерений (нечётная размерность)}
	\centering
	\begin{tabular}{|c|c|c|c|c|c|c|}
		\hline
		Размер n&&&&&&\\
		/    & 51 &101 & 201 & 301 & 401 & 501\\
		Алгоритм    &&&&&&\\
		\hline
		Стандартный & $5.4*10^{-4}$ & $4.2*10^{-3}$ & 0.037 & 0.133 & 0.329 & 0.838\\
		\hline
		Виноград & $4.2*10^{-4}$ & $0.003$ & 0.028 & 0.098 & 0.240 & 0.6\\
		\hline
		Виноград & $3.5*10^{-4}$ & $2.5*10^{-3}$ & 0.025 & 0.082 & 0.206 & 0.512\\
		(оптимизированный) &&&&&&\\
		\hline
	\end{tabular}
\end{table}