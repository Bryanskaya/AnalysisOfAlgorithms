\textbf{Характеристики ПО}

При проведении замеров времени использовался компьютер, имеющий следующие характеристики:
\begin{itemize}
	\item OC - Windows 10 Pro
	\item Процессор - Inter Core i7 10510U (1800 МГц)
	\item Объём ОЗУ - 16 Гб
\end{itemize}

\textbf{Измерения}

Для проведения замеров процессорного времени использовались квадратные матрицы. Их содержимое генерируется случайным образом. Было проведено две серии экспериментов, ориентированных на выявление чувствительности алгоритмов к чётным и нечётным значениям N.
\begin{itemize}
	\item[1)] ${50, 100, 200, 300, 400, 500, 600, 700}$
	\item[2)] ${51, 101, 201, 301, 401, 501, 601, 701}$
\end{itemize} 

Каждый замер проводится 5 раз для получения более точного среднего результата.

В \hyperref[table_4_1]{таблице 4.1} и \hyperref[table_4_2]{таблице 4.2} представлены результаты замеров процессорного времени работы реализаций алгоритмов (в сек).

\begin{table}[ph] \label{table_4_1}
	\caption{Результаты измерений (чётная размерность)}
	\centering
	\begin{tabular}{|c|c|c|c|c|c|c|c|c|}
		\hline
		Размер n&&&&&&&&\\
		/    &50 &100 & 200 & 300 & 400 & 500 & 600 & 700\\
		Алгоритм    &&&&&&&&\\
		\hline
		Стандартный & $5.1*10^{-4}$ & $4.1*10^{-3}$ & 0.037 & 0.133 & 0.322 & 0.777 & 1.08 & 1.445 \\
		\hline
		Виноград & $3.5*10^{-4}$ & $2.9*10^{-3}$ & 0.027 & 0.096 & 0.237 & 0.559 & 0.727 & 1.226\\
		\hline
		Виноград & $3.4*10^{-4}$ & $2.5*10^{-3}$ & 0.024 & 0.084 & 0.207 & 0.474 & 0.601 & 1.101\\
		(оптимизированный) &&&&&&&&\\
		\hline
	\end{tabular}
\end{table}

\begin{table}[ph] \label{table_4_2}
	\caption{Результаты измерений (нечётная размерность)}
	\centering
	\begin{tabular}{|c|c|c|c|c|c|c|c|c|}
		\hline
		Размер n&&&&&&&&\\
		/    & 51 &101 & 201 & 301 & 401 & 501 & 601 & 701\\
		Алгоритм    &&&&&&&&\\
		\hline
		Стандартный & $5.4*10^{-4}$ & $4.2*10^{-3}$ & 0.037 & 0.133 & 0.329 & 0.838 & 0.982 & 1.628\\
		\hline
		Виноград & $4.2*10^{-4}$ & $0.003$ & 0.028 & 0.098 & 0.240 & 0.6 & 0.852 & 1.465\\
		\hline
		Виноград & $3.5*10^{-4}$ & $2.5*10^{-3}$ & 0.025 & 0.082 & 0.206 & 0.512 & 0.705 & 1.052\\
		(оптимизированный) &&&&&&&&\\
		\hline
	\end{tabular}
\end{table}

Согласно полученным данным можно сделать следующие \textbf{выводы}.
\begin{itemize}
	\item[1)]Оптимизированный алгоритм Винограда осуществляет умножение матриц быстрее, чем два других сравниваемых алгоритма (на 36\% быстрее стандартного).
	\item[2)]Оптимизированный алгоритм Винограда показывает результаты, примерно на 14\%, чем неоптимизированный, что подтверждает разницу в рассчётах трудоёмкости обоих алгоритмов.
	\item[3)]Результаты на матрицах с чётным и нечётным размером N отличаются на предложенном множестве значений, но не существенно (около 9\%).
	\item[4)]Наихудшие результаты измерений показал стандартный алгоритм умножения.
	\end{itemize}