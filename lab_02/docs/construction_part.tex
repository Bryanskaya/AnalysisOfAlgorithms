Рассмотрим и оценим работу алгоритмов на матрицах $A[M \times N]$ и $B[N \times Q]$. 

\section{Стандартный алгоритм умножения матриц}
\qquadВ основе этого алгоритма лежит формула (\ref{formula1}). То есть для вычисления произведения двух матриц, каждая строка первой матрицы почленно умножается на каждый столбец второй, и затем подсчитывается сумма таких произведений, и полученный результат записывается в соответствующую ячейку результурующей матрицы.

\section{Алгоритм Винограда}
\qquadЦель данного алгоритма - сократить долю умножений в самом тяжёлом, затратном участке кода. Для этого используется формула (\ref{formula3}).\\

Некоторые из слагаемых можно вычислить заранее и использовать повторно для каждой строки первой матрицы и для каждого столбца второй. Таким образом, трудоёмкость алгоритма уменьшается за счёт сокращения количества производимых операций.\\

В этом алгоритме важно учитывать, что при нечётном значении $N$, необходимо вычислять дополнительное слагаемое $u_N \cdot v_N$.

\section{Требования к ПО}
\qquadДля корректной работы алгоритмов и проведения тестов необходимо выполнить следующее.
\begin{enumerate}
	\item[1)]Обеспечить возможность ввода двух матриц через консоль и выбора алгоритма для умножения.
	\item[2)]В случае ввода размеров матриц, не удовлетворяющих главному условию, вывести соответствующее сообщение. Программа не должна аварийно завершаться.
	\item[3)]Программа должна рассчитать искомую матрицу и вывести её на экран.
	\item[4)]Реализовать функцию замера процессорного времени, которое выбранный метод затрачивает на вычисление результата. Дать возможность пользователю ввести размер рассматриваемых матриц через консоль. Вывести результаты замеров на экран.
\end{enumerate}


\section{Заготовки тестов}
\qquadПри проверке на корректность работы реализованных функций необходимо провести следующие тесты:
\begin{itemize}
	\item умножение матриц размером $1 \times 1$;
	\item квадратные матрицы;
	\item прямоугольные матрицы;
	\item чётное и нечётное значение $N$.
\end{itemize}