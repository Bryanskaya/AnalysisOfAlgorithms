В ходе лабораторной работы была достигнута поставленная цель, а именно, оценена трудоёмкость алгоритмов умножения матриц и рассмотрены возможные оптимизации алгоритма Винограда.  \\

В процессе выполнения были решены все задачи. Описаны все рассматриваемые алгоритмы, дана теоретическая оценка трудоёмкости каждого. Все проработанные алгоритмы реализованы, кроме того, были проведены замеры процессорного времени работы на материале серии экспериментов и проведён сравнительный анализ, сделаны выводы:
\begin{itemize}
	\item[1)]оптимизированный алгоритм Винограда, как и ожидалось, выполняет умножение матриц быстрее, чем два других сравниваемых алгоритма (на 36\% он менее затратен по процессорному времени, чем стандартный и на 14\%, чем неоптимизированный);
	\item[2)]оптимизированный алгоритм Винограда осуществляет умножение мтариц быстрее, чем неоптимизированный, что подтверждает разницу в рассчётах трудоёмкости обоих алгоритмов;
	\item[3)]результаты отличаются несущественно на матрицах с чётным и нечётным размером N (разница примерно в 9\%);
	\item[4)]наихудшие результаты измерений времени показал стандартный алгоритм умножения.
\end{itemize}