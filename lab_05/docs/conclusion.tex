В ходе лабораторной работы была достигнута поставленная цель, а именно, разработан конвейер, работающий в параллельном режиме.\\

В процессе выполнения были решены все задачи. Описан принцип работы конвейера и всех рассматриваемых алгоритмов шифрования, лежащих в основе работы лент. Все проработанные алгоритмы реализованы. Кроме того, в ходе эксперимента, для каждой из задач были вычислены время ожидания в очереди и общее время обработки. Среди всех значений найдены максимальное, минимальное и среднее, и на основе полученных результатов сделаны выводы.\\

По результатам проведения эксперимента были сделаны следующие заключения.
\begin{itemize}
	\itemВремя ожидания в очереди не превышает 0.0037\% от времени выполнения.
	\itemВремя простоя в реализованном конвейере значительно меньше по сравнению со всей обработкой.
	\itemПроцесс обработки задач на лентах примерно одинаков по временным затратам, что удовлетворяет требованиям, и предотвращает вероятность простоя какой-либо ленты.
\end{itemize}