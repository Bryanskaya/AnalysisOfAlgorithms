В этом разделе будут поставлены цель и основные задачи лабораторной работы, которые будут решаться в ходе её выполнения.

\section{Цель и задачи}
\qquad\textbf{Цель} данной работы: получить навык разработки конвейера, работающего в параллельном режиме.\\

Для достижения поставленной цели необходимо решить ряд следующих \textbf{задач}:
\begin{enumerate}
	\item[1)] описать алгоритмы, на основе которых будет строиться работа конвейера;
	\item[2)] описать алгоритм работы конвейера;
	\item[3)] реализовать все рассмотренные алгоритмы;
	\item[4)] получить время ожидания в очереди и полное время решения для каждой задачи;
	\item[5)] среди найденных временных значений найти минимальное/максимальное/среднее время.
\end{enumerate}

В качестве основной работы берётся шифрование строк, которое разделено на три стадии, на каждой из которых представлен свой алгоритм.

\section{Основные требования к алгоритмам шифрования}
Сюда нужно отнести следующее.
\begin{itemize}
	\item Алгоритм должен быть надёжным, не допускать ситуации, когда ключ (если он используется) и сам принцип шифрования успешно угадывались сторонними лицами.
	\item Должен допускать эффективную программную реализацию.
	\item И быть достаточно простым для написания кода, чтобы минимизировать вероятность программных ошибок.
	\item Если алгоритм использует ключи, то любую случайную строку битов нужной длины, следует рассматривать в качестве возможного ключа.
	\item Используемый метод должен легко модифицироваться для различных уровней безопасности и удовлетворять как минимальным, так и максимальным требованиям.
\end{itemize}

\section{Используемые алгоритмы шифрования}

\subsection{Алгоритм, в основе которого лежит операция XOR}
\qquadОдин из самых простых алгоритмов шифрования. Он основан на применении бинарной логической операции исключающего или (XOR). \\

Операция XOR обладает симметричностью. Это значит, что если зашифровать одну и ту же строку 2 раза с одним и тем же ключом, то в результате получается эта же строка без изменений.

\subsection{Шифр Виженера}
\qquadЭтот метод используется для шифрования буквенного текста с использованием ключевого слова. В его основе лежит алгоритм Цезаря, но в отличие от него, он более надежный и безопасный. \\

Шифр можно записать с помощью формулы \ref{formula1}.
\begin{equation}\label{formula1}
	c_i = (m_i + k_i)\:mod\:n,
\end{equation}
где $m_i$ -- код $i-$ой буквы строки, которую нужно зашифровать, $k_i$ -- код буквы ключа, $n$ -- количество букв в алфавите, $c_i$ -- итоговое значение кода зашифрованной буквы, $mod$ -- операция получения остатка от деления. Далее по коду символа определяется соответсвующая ему буква \cite{cript}.

\subsection{Транспозиция}
\qquadЭтот алгоритм основан на перестановке символов в строке по какому-либо правилу. Для усложнения метода используется двойная перестановка, т.е. после первого прохода по строке, осуществляется второй проход, также меняющий местами символы, но уже по другому принципу.\\

Таким образом, не зная правил, по которым производилось шифрование достаточно сложно подобрать нужные. Задача усложняется тем, что неизвестно сколько и какие методы используются в конкретном случае \cite{cript2}

\section{Общие принципы работы конвейера}
\qquadПусть конвейер состоит из трёх лент. Тогда будем симулировать каждую из них, потоком, причём, на каждую ленту выделено по одному потоку.
Под потоком подразумевается непрерывнная часть кода процесса, которая может выполняться с другими частями выполняемой программы.\\

Пусть существует план -- обработать N объектов. Рабочий поток живёт до тех пор, пока не выполнит весь план.\\

Главный поток запускает рабочие потоки и выдаёт задачи (объекты) на обработку первому потоку. После того, как задача обработается на первой ленте, она передаётся на вторую, в это время как, на первую поступает другая задача. Аналогично организуется работа на втором и третьем потоках. Процессы на лентах выполняются параллельно.\\

Когда все задачи будут последовательно обработаны на трёх лентах конвейера, главный поток соберёт статистику и выведет её в интерфейс, либо файл.\\

Перед тем, как пройти стадию обработки на какой-либо ленте, каждая задача попадает в очередь из таких же задач. Максимально полезная работа контейенера возникает тогда, когда загружены все потоки, и время простоя минимально.

\section*{Вывод}
\addcontentsline{toc}{section}{Вывод}
\qquadБыл рассмотрен общий принцип работы конвейера, а также алгоритмы шифрования на каждой из его лент.