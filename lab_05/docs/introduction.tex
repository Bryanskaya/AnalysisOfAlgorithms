В этой лабораторной работе будет рассматриваться разработка конвейера, работающего в параллельном режиме. В качестве основной работы берётся шифрование строк, которое разделено на три стадии. \\

\textbf{Конвейеризация} (или конвейерная обработка) в общем случае основана на разделении подлежащей исполнению задачи на более мелкие части, называемые ступенями/стадиями, организовав передачу данных от одного этапа к следующему.  Производительность при этом возрастает благодаря тому, что одновременно на различных ступенях конвейера выполняются несколько задач. \cite{Conveyor}\\

Моделирование конвейерной обработки осуществляется с помощью потоков, на каждую ленту выделяется отдельный поток (или несколько, в зависимости от сложности решаемой задачи).\\

\textbf{Шифрование} (зашифрование) — процесс применения шифра к защищаемой информации, т.е. преобразование исходного сообщения в зашифрованное.\\

\textbf{Дешифрование} (расшифрование) — процесс, обратный шифрованию, т. е. преобразование шифрованного сообщения в исходное.\\

Под \textbf{шифром} понимается совокупность методов и способов обратимого преобразования информации с целью ее защиты от несанкционированного доступа (обеспечения конфиденциальности информации).\\

\textbf{Ключ} – это минимально необходимая информация (за исключением сообщения, алфавитов и алгоритма), для шифрования и дешифрования сообщений.\\

Под \textbf{алгоритмом} подразумевается набор правил (инструкций), определяющих содержание и порядок операций по шифрованию и дешифрованию информации.\\

В данной лабораторной работе будут использоваться такие алгоритмы шифрования, как усложненный метод с использованием операции XOR, алгоритм Виженера и метод транспозиции. 














